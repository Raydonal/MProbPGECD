\documentclass[12pt]{extarticle}
% \usepackage{fontspec}

% ==================================================

% document parameters
% \usepackage[spanish, mexico, es-lcroman]{babel}
\usepackage[english]{babel}
\usepackage[margin = 1in, includehead, 
headheight=30mm, 
a4paper,
total={170mm,257mm},
left=20mm,
top=5mm,
]{geometry}

% ==================================================

% Packages for math
\usepackage{mathrsfs}
\usepackage{amsfonts}
\usepackage{amsmath}
\usepackage{amsthm}
\usepackage{amssymb}
\usepackage{physics}
\usepackage{dsfont}
\usepackage{esint}

% ==================================================

% Packages for writing
\usepackage{enumerate}
\usepackage[shortlabels]{enumitem}
\usepackage{framed}
\usepackage{csquotes}

% ==================================================

% Miscellaneous packages
\usepackage{float}
\usepackage{tabularx}
\usepackage[dvipsnames]{xcolor}
\usepackage{multicol}
\usepackage{subcaption}
\usepackage{caption}
\captionsetup{format = hang, margin = 10pt, font = small, labelfont = bf}

% Citation
\usepackage[round, authoryear]{natbib}

% Hyperlinks setup
\usepackage{hyperref}
\definecolor{links}{rgb}{0.36,0.54,0.66}
\hypersetup{
   colorlinks = true,
    linkcolor = black,
     urlcolor = blue,
    citecolor = blue,
    filecolor = blue,
    pdfauthor = {Author},
     pdftitle = {Title},
   pdfsubject = {subject},
  pdfkeywords = {one, two},
  pdfproducer = {LaTeX},
   pdfcreator = {pdfLaTeX},
   }
   
 \RequirePackage{boxedminipage}  
\definecolor{darkred}{rgb}{0.55, 0.0, 0.0}
 \RequirePackage{layout,color}%,enum}%,verbatim}  
   \definecolor{gray50}{gray}{.5}
   \definecolor{gray40}{gray}{.6}
   \definecolor{gray30}{gray}{.7}
   \definecolor{gray20}{gray}{.8}
   \definecolor{gray10}{gray}{.9}
   \definecolor{gray05}{gray}{.95}

   \usepackage{fancyhdr}
   
   
   
   \pagestyle{fancy}
   
   \fancyhead[C]{
   	%	\vspace{3cm}
   	%	\rput(0,2){
   		\begin{minipage}{1.75cm}\centering
   			\includegraphics[width=1.45cm]{Figuras/ufbaBlack.eps}
   		\end{minipage}~\hfill~\begin{minipage}{10cm}\large\centering
   						\textcolor{darkred}{
   			{\scshape\bfseries\footnotesize UNIVERSIDADE FEDERAL DA BAHIA}\\
   			{\scshape\bfseries\footnotesize INSTITUTO DE MATEMÁTICA E ESTATÍSTICA}\\
   			{\scshape\bfseries\footnotesize \medskip PROGRAMA DE PÓS-GRADUAÇÃO EM ESTATÍSTICA E CIÊNCIA DE DADOS}\\[20pt]}
   			% Aqui as informações da prova e a data
   			%			{\normalsize\scshape  prova escrita - processo seletivo - mestrado\\ { 16/05/2024}} 
   		\end{minipage}~\hfill~\begin{minipage}{1.75cm}\centering
   			\includegraphics[width=2.0cm]{Figuras/ufba.eps}
   		\end{minipage}
   		%}
   }
   \fancyfoot[C]{\thepage} 
\input{Config/format}
% --- Basic commands ---
%   Euler's constant
\newcommand{\eu}{\mathrm{e}}

%   Imaginary unit
\newcommand{\im}{\mathrm{i}}

%   Sexagesimal degree symbol
\newcommand{\grado}{\,^{\circ}}

% --- Comandos para álgebra lineal ---
% Matrix transpose
\newcommand{\transpose}[1]{{#1}^{\mathsf{T}}}

%%% Comandos para cálculo
%   Definite integral from -\infty to +\infty
\newcommand{\Int}{\int\limits_{-\infty}^{\infty}}

%   Indefinite integral
\newcommand{\rint}[2]{\int{#1}\dd{#2}}

%  Definite integral
\newcommand{\Rint}[4]{\int\limits_{#1}^{#2}{#3}\dd{#4}}

%   Dot product symbol (use the command \bigcdot)
\makeatletter
\newcommand*\bigcdot{\mathpalette\bigcdot@{.5}}
\newcommand*\bigcdot@[2]{\mathbin{\vcenter{\hbox{\scalebox{#2}{$\m@th#1\bullet$}}}}}
\makeatother

%   Hamiltonian
\newcommand{\Ham}{\hat{\mathcal{H}}}

%   Trace
\renewcommand{\Tr}{\mathrm{Tr}}

% Christoffel symbol of the second kind
\newcommand{\christoffelsecond}[4]{\dfrac{1}{2}g^{#3 #4}(\partial_{#1} g_{#2 #4} + \partial_{#2} g_{#1 #4} - \partial_{#4} g_{#1 #2})}

% Riemann curvature tensor
\newcommand{\riemanncurvature}[5]{\partial_{#3} \Gamma_{#4 #2}^{#1} - \partial_{#4} \Gamma_{#3 #2}^{#1} + \Gamma_{#3 #5}^{#1} \Gamma_{#4 #2}^{#5} - \Gamma_{#4 #5}^{#1} \Gamma_{#3 #2}^{#5}}

% Covariant Riemann curvature tensor
\newcommand{\covariantriemanncurvature}[5]{g_{#1 #5} R^{#5}{}_{#2 #3 #4}}

% Ricci tensor
\newcommand{\riccitensor}[5]{g_{#1 #5} R^{#5}{}_{#2 #3 #4}}


\def\stacksymbols #1#2#3#4{\def\theguybelow{#2}
	\def\verticalposition{\lower#3pt}
	\def\spacingwithinsymbol{\baselineskip0pt\lineskip#4pt}
	\mathrel{\mathpalette\intermediary#1}}
\def\intermediary#1#2{\verticalposition\vbox{\spacingwithinsymbol
		\everycr={}\tabskip0pt
		\halign{$\mathsurround0pt#1\hfil##\hfil$\crcr#2\crcr
			\theguybelow\crcr}}}
		
\def\xtil{\stacksymbols{X}{\sim}{5}{.7}}
\def\thetatil{\stacksymbols{\theta}{\sim}{5}{.7}} 
\def\xtill{\stacksymbols{x}{\sim}{5}{.7}}
\def\ytil{\stacksymbols{Y}{\sim}{5}{.7}}
\def\ytill{\stacksymbols{y}{\sim}{5}{.7}}
\def\ttil{\stacksymbols{T}{\sim}{5}{.7}}
\def\ttill{\stacksymbols{t}{\sim}{5}{.7}}

\oddsidemargin -0.6in %distancia da margin esquerda
\evensidemargin 1.0in
\textwidth 7.5in %largo do documento
\headheight 1.0in 
\topmargin -0.7in %control superio da margin
\textheight 10.8in %comprimento do documento
%\footheight 1.0in 

% 
\begin{document}
	
	\footnotesize

{\scshape\bf\textcolor{darkred}{PROVA III - Probabilidade (PPGECD000000001)} - 24/07/2025 }



\textsf{\textbf{Professor:}} \text{Raydonal Ospina Martinez.} 
% \quad {\bf\small E-mail:} \href{mailto:raydonal@castlab.org}{\texttt{raydonal@castlab.org}}

\bigskip 
\begin{regras}
Leia com atenção as perguntas. {\bf Todas as resoluções devem ser DETALHADAS.}  Seja claro e organizado. 
\end{regras}
\vspace{2ex}

	
\begin{problema}{}{pr-1}
Seja $X$ uma variável aleatória com distribuição de probabilidade $f(x)=P(X=x)$ e seja $A\in {\cal B}(\mathbb{R})$ um conjunto de Borel em $\mathbb{R}$ tal que $p=P(X\in A)>0.$ Definimos a função de distribuição de probabilidade condicional de $X$ dado o evento $(X\in A)$ como
$$
f(x|A) = \frac{1}{p}f(x)\mathbb{I}_A(x) = \begin{cases}
	\frac{1}{p}f(x),  & \text{se} \ x\in A, \\
	0, & \text{caso contrário.}
\end{cases}
$$
Para $f(x)$  definida por
	$$
	f(x)=P(X=x) = \begin{cases}
		\frac{1}{2n+1}, & \text{se} \quad x \in {\cal M}, \\
		0 & \text{caso contrário},
	\end{cases}
	$$ em que ${\cal M}=\{-n,-(n-1), \ldots, -2,-1, 0, 1, 2, \ldots, (n-1), n \} $ e $A=\{0,1,2,3, \ldots , n-1, n\}.$ 
Compare $f(x)$ e $f(x|A)$ em termos dos valores esperados $E(aX+b)$ e $E(aX+b|A),$ com $a,b \in \mathbb{R}$ constantes conhecidas, respectivamente. 

{\bf Dica:} 
$\sum^n_{i = 1} i = 1 + 2 + 3 +\cdots + n = \frac{n ( n + 1 )}{2}$ e $
\sum^n_{i = m} 1 = n+1-m.$	

\end{problema}



\begin{problema}{}{pr-2}
Seja $X$ uma variável aleatória uniforme no intervalo (0,1) e $Y=X^2.$ Calcular o coeficiente de correlação $\rho(X, Y).$ São $X$ e $Y$ independentes? Explique.
\end{problema}


\begin{problema}{}{pr-3}
%Seja $X$ uma variável aleatória e seja $\varphi : \mathbb{R} \to [0, \infty)$ uma função convexa e monótona não decrescente tal que $E(\varphi(X)) < \infty$. Demonstre que para qualquer constante $\epsilon > 0$,
%\[
%P(X \ge \epsilon) \le \frac{E(\varphi(X))}{\varphi(\epsilon)}.
%\]
Seja $X$ uma variável aleatória \textbf{não negativa} ($P(X \ge 0)=1$) e seja $\varphi : [0, \infty) \to [0, \infty)$ uma função \textbf{convexa} e \textbf{estritamente crescente}. Demonstre que para qualquer constante $\epsilon > 0$:
\[
P(X \ge \epsilon) \le \frac{\varphi^{-1}(E[\varphi(X)])}{\epsilon}
\] em que $\varphi^{-1}$ é a função inversa de $\varphi$.
\end{problema}


\begin{problema}{}{pr-4}
Suponha que as sequências de variáveis aleatórias $\{X_n\}_{n\geq 1}$ e $\{Y_n\}_{n\geq 1}$ são tais que $X_n \xrightarrow{p} x$ e $Y_n \xrightarrow{p} y,$  em que $x$ e $y$ são dois números reais fixos. Demonstre que:
\[ X_n + Y_n \xrightarrow{p} x + y. \] Aqui, ($ \xrightarrow{p}$) indica convergência em probabilidade.
\end{problema}




\begin{problema}{}{pr5}
Seja $X$ com distribuição uniforme discreta no conjunto $\{0, 1\}$. Demonstre que a seguinte sucessão de variáveis aleatórias converge em distribuição, mas não converge em probabilidade.
\[
X_n = \begin{cases} 
	X & \text{se } n \text{ é par,} \\
	1-X & \text{se } n \text{ é ímpar.}
\end{cases}
\]
\end{problema}


\centering {\scshape\bf\textcolor{darkred}{BOA PROVA}}

%

\end{document}


%
%\begin{problema}{}{pr3}
%	Seja $X$ uma vari\'avel aleat\'oria com distribui\c{c}\~ao gama sob uma nova reparametriza\c{c}\~ao, aqui denotada por $X\sim G(\mu, \phi)$,
%	tal que
%	$$f(x;\mu, \phi) = {1\over\Gamma(\phi)}\left( {\phi x}\over{\mu}\right)^\phi\exp\left(-{{\phi x}\over{\mu}}\right){1\over x},\,\, x \geq 0, \mu >0, \phi> 0 \ \ \text{e} \ \ \Gamma(\phi) = \int^\infty_0 t^{(\phi - 1)}e^{-t} dt .$$  
%	\begin{enumerate}
%		\item Mostre que $f(x;\mu, \phi)$ forma uma fam\'{\i}lia exponencial bidimensional.
%		\item  Mostre que,  $$\mu = {\rm E}(X)\quad{\rm e}\quad \phi^{-1/2} = { \sqrt{{\rm Var}(X)}\over{\rm E}(X) }.$$
%		\item  Sejam  $X_1,\ldots,X_n$ uma amostra aleat\'oria de $X\sim G(\mu,\phi_1)$ 
%		e $Y_1,\ldots, Y_n$ uma amostra aleat\'oria de   $Y\sim G(\mu,\phi_2)$. Aqui,
%		$X$ e $Y$ s\~ao popula\c{c}\~oes independentes, 
%		$X \in \mathbb{R}^+$, $Y\in \mathbb{R}^+$, $\mu \in  \mathbb{R}^+$,  $\phi_1 \in \mathbb{R}^+$ e $\phi_2 \in \mathbb{R}^+.$
%		Determine a densidade conjunta $f_{(\xtill,\ytill)}{(\mu, \phi_1,\phi_2)}$.
%		\item  Mostre que a estat\'{\i}stica $$T =\Bigg(\sum_{i=1}^n X_i, \sum_{i=1}^n Y_i, \sum_{i=1}^n \log X_i, \sum_{i=1}^n\log Y_i \Bigg)$$ n\~ao \'e completa para a fam\'{\i}lia definida em 3.
%		\item O que acontece com as análises anteriores se as distribuições gama aqui consideradas são curvas, i.e. quando $\mu=\phi_1=\phi_2$?
%	\end{enumerate}
%\end{problema}	
%
%\begin{problema}{}{pr-4}
%	Seja $\hat{\theta}$ um estimador não viesado de um parâmetro escalar $\theta \in \Theta_\theta \equiv \mathbb{R}$, satisfazendo $E_\theta(\hat{\theta}^2) < \infty$ para todo $\theta \in \Theta_\theta$. Dizemos que $\hat{\theta}$ é um estimador de variância mínima uniforme não viesado (UMVU, do inglês \textit{Uniform Minimum Variance Unbiased}) se 
%	\[
%	\text{Var}_\theta(\hat{\theta}) \leq \text{Var}_\theta(\tilde{\theta})
%	\]
%	para todo $\theta \in \Theta_\theta$ e qualquer outro estimador não viesado $\tilde{\theta}$.  Considere o estimador $\hat{\theta}_\tau = \hat{\theta} + \tau U$, ou outro equivalente. Mostre que uma condição necessária para que $\hat{\theta}$ seja um estimador UMVU é que 
%	\[
%	E_\theta(\hat{\theta}U) = 0,
%	\]
%	para todo $\theta \in \Theta_\theta$ e para todos os estimadores $U$ com $E_\theta(U) = 0$ e $E_\theta(U^2) < \infty$. Assim, $\hat{\theta}$ deve ser "não correlacionado com todo estimador não viesado de zero".
%	
%	Agora, para $n \geq 2$, sejam $X_1, \dots, X_n$ variáveis independentes e identicamente distribuídas com distribuição uniforme $U(\theta, 2\theta)$, para algum $\theta > 0$. 
%	
%	\begin{enumerate}
%		\item Mostre que $\tilde{\theta} = \frac{3}{2} X_1$ é um estimador não viesado de $\theta$.
%		\item Encontre um estimador não viesado $\hat{\theta}$ que seja função de uma estatística suficiente mínima e que satisfaça 
%		\[
%		\text{Var}_\theta(\hat{\theta}) < \text{Var}_\theta(\tilde{\theta}),
%		\]
%		para todo $\theta > 0$.
%		\item Verifique se $\hat{\theta}$ é UMVU.
%	\end{enumerate}
%\end{problema}	
%
%\begin{problema}{}{pr-5}
%%Seja \( S \) uma estatística suficiente completa para um parâmetro \( \theta \in \Theta_\theta \), e seja \( C \) uma estatística constante em distribuição, i.e. uma estatística cuja distribuição não depende do parâmetro desconhecido $\theta$ do modelo. Mostre que \( S \) e \( C \) são independentes para cada \( \theta \in \Omega_\theta \). Qual é o significado desse resultado para a inferência estatística sobre \( \theta \)?
%Seja  $X_1,\ldots
%,X_n$ uma amostra aleat\'oria de 
%$$f(x;\theta) = { 1 \over \theta_2}e^{-{(x -\theta_1)\over \theta_2 }}, \,\, \theta_1 \leq x <+\infty,\,\,\,
%-\infty < \theta_1 < +\infty, \,\, 0 < \theta_2 < +\infty.$$
%\begin{enumerate}
%	
%	\item Prove que esta \'e uma fam\'{\i}lia de loca\c{c}\~ao e escala.
%	\item Mostre que
%	$$V = {{X_{(n)} - X_{(1)}}\over{\sqrt{\sum_{i=1}^n(X_i -\overline X)^2}}}$$
%	\'e ancilar para a fam\'{\i}lia.
%	\item  Sejam 
%	$$Z = {X_{(n)} - X_{(1)}} \,\, {\rm e}\,\,\,T = {X_{(n)}}.$$ 
%	Prove que $Z$ e $T$ s\~ao vari\'aveis aleat\'orias independentes.
%	Aqui, $$X_{(n)} =\max\{X_1, \ldots,X_n \}   \quad \text{e} \quad X_{(1)} = \min\{X_1, \ldots,X_n \}$$ 
%\end{enumerate}
%\end{problema}	
%%\bibliography{references}
%

%\end{document}
%
%
%Abaixo está a solução detalhada para cada parte do problema:
%
%---
%
%### **Problema:**
%Bactérias são distribuídas aleatoriamente em um fluido com densidade média \(\theta\) por unidade de volume (\(\theta \in H \subseteq [0, \infty)\)). A probabilidade de nenhuma bactéria estar presente em um volume \(v\) é:
%\[
%P_\theta(\text{nenhuma bactéria no volume } v) = e^{-\theta v}.
%\]
%O objetivo é decidir se há ou não bactérias no fluido, com base em um teste aplicado a uma amostra de volume \(v\). Uma decisão incorreta resulta em perda de \(1\), enquanto uma decisão correta não implica perda.
%
%---
%
%### **Parte (i): \(H = [0, \infty)\)**
%
%#### Regras de decisão não randomizadas:
%Uma regra de decisão não randomizada \(d\) decide entre:
%- \(d = 1\): Aceitar que há bactérias (\(\theta > 0\)).
%- \(d = 0\): Aceitar que não há bactérias (\(\theta = 0\)).
%
%A decisão é baseada no resultado do teste aplicado ao volume \(v\):
%- Se a amostra contiver bactérias, decidimos \(d = 1\).
%- Se a amostra não contiver bactérias, decidimos \(d = 0\).
%
%#### Funções de risco:
%A função de risco \(R(\theta, d)\) mede a perda esperada ao aplicar a regra \(d\):
%- Se \(d = 1\), a perda ocorre apenas se \(\theta = 0\), com probabilidade \(e^{-\theta v}\).
%- Se \(d = 0\), a perda ocorre se \(\theta > 0\), com probabilidade \(1 - e^{-\theta v}\).
%
%As funções de risco são:
%\[
%R(\theta, d) =
%\begin{cases}
%	1 - e^{-\theta v}, & \text{se \(d = 0\)}, \\
%	e^{-\theta v}, & \text{se \(d = 1\)}.
%\end{cases}
%\]
%
%#### Regras admissíveis:
%Uma regra de decisão \(d\) é admissível se não existe outra regra com menor risco para todos os valores de \(\theta\). Ambas as regras \(d = 0\) e \(d = 1\) são admissíveis neste caso, pois minimizam o risco para intervalos complementares de \(\theta\).
%
%---
%
%### **Parte (ii): \(H = \{0, 1\}\)**
%
%#### Conjunto de risco \(S\):
%Se \(H = \{0, 1\}\), o conjunto de risco é:
%\[
%S = \{(R(0, d), R(1, d)) : d \text{ é uma regra randomizada}\}.
%\]
%
%Uma regra randomizada pode ser definida como \(d(X) = \alpha\), onde \(0 \leq \alpha \leq 1\) é a probabilidade de decidir \(d = 1\).
%
%Os riscos são:
%\[
%R(0, d) = \alpha,
%\]
%\[
%R(1, d) = 1 - \alpha e^{-v}.
%\]
%
%O conjunto de risco \(S\) contém todos os pares \((\alpha, 1 - \alpha e^{-v})\) para \(0 \leq \alpha \leq 1\).
%
%#### Regra minimax:
%A regra minimax minimiza a máxima perda:
%\[
%\min_\alpha \max \{R(0, d), R(1, d)\}.
%\]
%
%Resolva:
%\[
%\alpha = \frac{1}{1 + e^{-v}}.
%\]
%
%Assim, a regra minimax é \(d(X) = \frac{1}{1 + e^{-v}}\).
%
%---
%
%### **Parte (iii): \(H = [0, \infty)\), com prior \(\pi\):**
%
%#### Priori:
%\[
%\pi(\{0\}) = \frac{1}{3}, \quad \pi(A) = \frac{2}{3} \int_A e^{-\theta} \, d\theta, \quad A \subseteq (0, \infty).
%\]
%
%#### Regras de decisão de Bayes:
%A regra de Bayes minimiza o risco esperado sob a priori:
%\[
%R_\pi(d) = \frac{1}{3} R(0, d) + \frac{2}{3} \int_0^\infty R(\theta, d) e^{-\theta} \, d\theta.
%\]
%
%Para \(d = 1\):
%\[
%R_\pi(1) = \frac{1}{3} e^{-v}.
%\]
%
%Para \(d = 0\):
%\[
%R_\pi(0) = \frac{2}{3} \int_0^\infty (1 - e^{-\theta v}) e^{-\theta} \, d\theta.
%\]
%
%Calcule a integral:
%\[
%\int_0^\infty e^{-\theta v} e^{-\theta} \, d\theta = \int_0^\infty e^{-\theta (v+1)} \, d\theta = \frac{1}{v+1}.
%\]
%
%Substitua para obter:
%\[
%R_\pi(0) = \frac{2}{3} \left(1 - \frac{1}{v+1}\right) = \frac{2}{3} \frac{v}{v+1}.
%\]
%
%Compare os riscos e escolha a melhor regra.
%
%#### Risco de Bayes:
%Substitua a regra de Bayes na função de risco para obter o risco correspondente.
%
%---
%
%### **Parte (iv): Custo \(v/24\):**
%
%O custo total é:
%\[
%\text{Custo total} = R_\pi(d) + \frac{v}{24}.
%\]
%
%Otimize \(v\) resolvendo:
%\[
%\frac{\partial}{\partial v} \left(R_\pi(d) + \frac{v}{24}\right) = 0.
%\]
%
%Repita o cálculo para o custo por unidade de volume \(1/6\).
%
%---
%
%Se precisar de ajuda com os cálculos detalhados ou dúvidas específicas, avise! 
%
%
%### **Parte I - Encontrar as Regras Admissíveis**
%
%#### Problema:
%Estamos interessados em determinar todas as regras de decisão não randomizadas e avaliar quais delas são admissíveis.
%
%1. **Regras Não Randomizadas:**
%As regras de decisão não randomizadas para este problema são baseadas nos resultados do teste realizado sobre o volume \(v\). Essas regras \(d\) podem ser definidas como:
%\[
%d = 1 \quad (\text{decidir que há bactérias/células cancerígenas}),
%\]
%ou
%\[
%d = 0 \quad (\text{decidir que não há bactérias/células cancerígenas}).
%\]
%
%2. **Funções de Risco:**
%A função de risco \(R(\theta, d)\) mede a perda esperada para cada decisão \(d\):
%- Para \(d = 0\) (decidir que não há bactérias):
%\[
%R(\theta, 0) = \begin{cases}
%	0, & \text{se } \theta = 0, \\
%	1 - e^{-\theta v}, & \text{se } \theta > 0.
%\end{cases}
%\]
%- Para \(d = 1\) (decidir que há bactérias):
%\[
%R(\theta, 1) = \begin{cases}
%	1, & \text{se } \theta = 0, \\
%	e^{-\theta v}, & \text{se } \theta > 0.
%\end{cases}
%\]
%
%3. **Comparando os Riscos:**
%As regras admissíveis são aquelas que minimizam o risco \(R(\theta, d)\) para algum valor de \(\theta\) sem serem estritamente dominadas por outra regra.
%
%- Quando \(\theta = 0\), a decisão \(d = 0\) (nenhuma célula presente) tem risco \(R(0, 0) = 0\), enquanto \(R(0, 1) = 1\). Assim, \(d = 0\) é preferível neste caso.
%- Quando \(\theta > 0\), a decisão \(d = 1\) (células presentes) reduz o risco para \(R(\theta, 1) = e^{-\theta v}\), enquanto \(R(\theta, 0) = 1 - e^{-\theta v}\). Assim, \(d = 1\) é preferível para valores maiores de \(\theta\).
%
%4. **Conclusão:**
%- \(d = 0\) é admissível, pois minimiza o risco para \(\theta = 0\).
%- \(d = 1\) é admissível, pois minimiza o risco para \(\theta > 0\).
%- Nenhuma das regras é estritamente dominante em todo o domínio de \(\theta\).
%
%---
%
%### **Parte IV - Determinar o Volume Ótimo para Testar**
%
%#### Problema:
%Queremos encontrar o volume \(v\) que minimiza o custo total, que inclui o risco esperado e o custo do teste. 
%
%1. **Custo Total:**
%O custo total é dado por:
%\[
%\text{Custo Total}(v) = R_\pi(d, v) + \frac{v}{24},
%\]
%onde \(R_\pi(d, v)\) é o risco esperado sob a prior \(\pi\).
%
%#### Calculando o Risco Esperado \(R_\pi(d, v)\):
%De acordo com a prior fornecida:
%\[
%\pi(\{0\}) = \frac{1}{3}, \quad \pi(A) = \frac{2}{3} \int_A e^{-\theta} \, d\theta.
%\]
%
%Para a regra \(d = 1\):
%\[
%R_\pi(1, v) = \frac{1}{3} R(0, 1) + \frac{2}{3} \int_0^\infty R(\theta, 1) e^{-\theta} \, d\theta.
%\]
%Substituímos:
%\[
%R(0, 1) = 1, \quad R(\theta, 1) = e^{-\theta v}.
%\]
%
%\[
%R_\pi(1, v) = \frac{1}{3} \cdot 1 + \frac{2}{3} \int_0^\infty e^{-\theta v} e^{-\theta} \, d\theta.
%\]
%
%A integral é:
%\[
%\int_0^\infty e^{-\theta (v+1)} \, d\theta = \frac{1}{v+1}.
%\]
%
%Assim:
%\[
%R_\pi(1, v) = \frac{1}{3} + \frac{2}{3} \cdot \frac{1}{v+1}.
%\]
%
%Para a regra \(d = 0\), calcule:
%\[
%R_\pi(0, v) = \frac{1}{3} R(0, 0) + \frac{2}{3} \int_0^\infty R(\theta, 0) e^{-\theta} \, d\theta.
%\]
%
%Similarmente, resolvemos e determinamos \(R_\pi(0, v)\).
%
%#### Minimização:
%Agora, o custo total é:
%\[
%\text{Custo Total}(v) = \frac{1}{3} + \frac{2}{3} \cdot \frac{1}{v+1} + \frac{v}{24}.
%\]
%
%Derivamos em relação a \(v\):
%\[
%\frac{\partial}{\partial v} \left(\frac{1}{3} + \frac{2}{3} \cdot \frac{1}{v+1} + \frac{v}{24}\right) = 0.
%\]
%
%A derivada é:
%\[
%-\frac{2}{3} \cdot \frac{1}{(v+1)^2} + \frac{1}{24} = 0.
%\]
%
%Multiplicamos por \(24(v+1)^2\) para simplificar:
%\[
%-16 + (v+1)^2 = 0.
%\]
%
%Resolvemos:
%\[
%(v+1)^2 = 16 \quad \Rightarrow \quad v+1 = 4 \quad \Rightarrow \quad v = 3.
%\]
%
%#### Resposta para Custo \(v/24\):
%O volume ótimo para testar é \(v = 3\).
%
%#### Para o Custo \(1/6\) por Unidade de Volume:
%Se o custo for \(v/6\), o termo \(\frac{v}{24}\) muda para \(\frac{v}{6}\). Repita o processo para obter o novo volume ótimo.
%
%---
%
%Se precisar de mais explicações ou detalhes, estarei à disposição! 😊