% Options for packages loaded elsewhere
\PassOptionsToPackage{unicode}{hyperref}
\PassOptionsToPackage{hyphens}{url}
%
\documentclass[
]{article}
\usepackage{amsmath,amssymb}
\usepackage{iftex}
\ifPDFTeX
  \usepackage[T1]{fontenc}
  \usepackage[utf8]{inputenc}
  \usepackage{textcomp} % provide euro and other symbols
\else % if luatex or xetex
  \usepackage{unicode-math} % this also loads fontspec
  \defaultfontfeatures{Scale=MatchLowercase}
  \defaultfontfeatures[\rmfamily]{Ligatures=TeX,Scale=1}
\fi
\usepackage{lmodern}
\ifPDFTeX\else
  % xetex/luatex font selection
\fi
% Use upquote if available, for straight quotes in verbatim environments
\IfFileExists{upquote.sty}{\usepackage{upquote}}{}
\IfFileExists{microtype.sty}{% use microtype if available
  \usepackage[]{microtype}
  \UseMicrotypeSet[protrusion]{basicmath} % disable protrusion for tt fonts
}{}
\makeatletter
\@ifundefined{KOMAClassName}{% if non-KOMA class
  \IfFileExists{parskip.sty}{%
    \usepackage{parskip}
  }{% else
    \setlength{\parindent}{0pt}
    \setlength{\parskip}{6pt plus 2pt minus 1pt}}
}{% if KOMA class
  \KOMAoptions{parskip=half}}
\makeatother
\usepackage{xcolor}
\usepackage[margin=1in]{geometry}
\usepackage{graphicx}
\makeatletter
\def\maxwidth{\ifdim\Gin@nat@width>\linewidth\linewidth\else\Gin@nat@width\fi}
\def\maxheight{\ifdim\Gin@nat@height>\textheight\textheight\else\Gin@nat@height\fi}
\makeatother
% Scale images if necessary, so that they will not overflow the page
% margins by default, and it is still possible to overwrite the defaults
% using explicit options in \includegraphics[width, height, ...]{}
\setkeys{Gin}{width=\maxwidth,height=\maxheight,keepaspectratio}
% Set default figure placement to htbp
\makeatletter
\def\fps@figure{htbp}
\makeatother
\setlength{\emergencystretch}{3em} % prevent overfull lines
\providecommand{\tightlist}{%
  \setlength{\itemsep}{0pt}\setlength{\parskip}{0pt}}
\setcounter{secnumdepth}{5}
\usepackage[brazil]{babel}
\usepackage{amsmath}
\usepackage{amssymb}
\usepackage{amsthm}
\usepackage{geometry}
\usepackage{booktabs}
\usepackage{diagbox}
\ifLuaTeX
  \usepackage{selnolig}  % disable illegal ligatures
\fi
\usepackage{bookmark}
\IfFileExists{xurl.sty}{\usepackage{xurl}}{} % add URL line breaks if available
\urlstyle{same}
\hypersetup{
  pdftitle={Resolução Detalhada da Prova de Probabilidade},
  pdfauthor={Análise Estatística e Matemática},
  hidelinks,
  pdfcreator={LaTeX via pandoc}}

\title{Resolução Detalhada da Prova de Probabilidade}
\author{Análise Estatística e Matemática}
\date{06 de julho de 2025}

\begin{document}
\maketitle

{
\setcounter{tocdepth}{2}
\tableofcontents
}
\newtheorem{problema}{Problema}
\renewcommand{\theproblema}{\arabic{problema}}

\section*{Introdução}

Este documento apresenta a resolução detalhada dos problemas propostos
na avaliação. Cada questão é abordada com rigor matemático e
estatístico, explicitando todas as etapas do raciocínio, desde a
definição dos conceitos fundamentais até a obtenção do resultado final.
A notação e a linguagem utilizadas são consistentes com a teoria de
probabilidades e a estatística matemática, visando a clareza e a
precisão técnica.

\begin{problema}{}{pr-1}
    Seja $\Omega = \{a, b, c\}$  um espaço amostral, ${\cal F}={\cal P}(\Omega)$ o conjunto de partes de $\Omega$ como sua $\sigma$-álgebra e $P(\{\omega\})=\frac{1}{3}$ para todo $\omega \in \Omega.$ Consideremos as variáveis aleatórias $X$ e $Y$ definidas em $(\Omega, {\cal F}, P)$ como
    $$
    X(\omega) =
    \begin{cases}
        1, &  \text{se}  \quad  \omega=a, \ \text{ou} \ \omega = b,  \\
        0, &  \text{se}  \quad  \omega=c
    \end{cases} 
    \qquad \text{e} \qquad 
    Y(\omega) =
    \begin{cases}
        \ \pi, &  \text{se}  \quad  \omega=a, \\
        \ \frac{1}{2}, &  \text{se}  \quad  \omega=b, \\
        -1, &  \text{se}  \quad  \omega=c
    \end{cases} 
    $$
    Obtenha as distribuições condicionais acumuladas $F(X|Y)$ e $F(Y|X)$.
\end{problema}

\subsection*{Resolução do Problema 1}

Para obter as funções de distribuição acumulada (FDA) condicionais,
\(F(x|y) = P(X \le x | Y=y)\) e \(F(y|x) = P(Y \le y | X=x)\), o
primeiro passo é caracterizar a distribuição de probabilidade conjunta
do par de variáveis aleatórias \((X, Y)\).

\subsubsection*{1. Determinação da Função de Probabilidade Conjunta}

A função de probabilidade conjunta, \(p(x, y) = P(X=x, Y=y)\), é
determinada avaliando os pares \((X(\omega), Y(\omega))\) para cada
\(\omega \in \Omega\) e sua respectiva probabilidade. Dado que
\(P(\{a\}) = P(\{b\}) = P(\{c\}) = 1/3\):

\begin{itemize}
    \item Para $\omega = a$: Temos $X(a) = 1$ e $Y(a) = \pi$. A probabilidade deste evento é $P(\{a\}) = 1/3$. Portanto, $p(1, \pi) = P(X=1, Y=\pi) = 1/3$.
    
    \item Para $\omega = b$: Temos $X(b) = 1$ e $Y(b) = 1/2$. A probabilidade deste evento é $P(\{b\}) = 1/3$. Portanto, $p(1, 1/2) = P(X=1, Y=1/2) = 1/3$.
    
    \item Para $\omega = c$: Temos $X(c) = 0$ e $Y(c) = -1$. A probabilidade deste evento é $P(\{c\}) = 1/3$. Portanto, $p(0, -1) = P(X=0, Y=-1) = 1/3$.
\end{itemize}

Para todos os outros pares \((x,y)\), a probabilidade conjunta é zero. A
distribuição conjunta pode ser resumida na seguinte tabela, que também
inclui as distribuições marginais (soma das linhas e colunas).

\begin{center}
    \begin{tabular}{c|ccc|c}
        \toprule
        \diagbox[height=1.2cm]{$X$}{$Y$} & -1 & 1/2 & $\pi$ & $p_X(x)$ \\
        \midrule
        0 & 1/3 & 0 & 0 & 1/3 \\
        1 & 0 & 1/3 & 1/3 & 2/3 \\
        \midrule
        $p_Y(y)$ & 1/3 & 1/3 & 1/3 & 1 \\
        \bottomrule
    \end{tabular}
\end{center}

\subsubsection*{2. Determinação das Funções de Probabilidade Condicionais}

A função de probabilidade condicional é dada por
\(p(x|y) = \frac{p(x,y)}{p_Y(y)}\) e \(p(y|x) = \frac{p(x,y)}{p_X(x)}\).

\paragraph{Condicional de $X$ dado $Y=y$:}
\begin{itemize}
    \item Se $y=-1$: $p(x|-1) = \frac{p(x,-1)}{1/3}$. Assim, $p(0|-1)=1$ e $p(1|-1)=0$.
    \item Se $y=1/2$: $p(x|1/2) = \frac{p(x,1/2)}{1/3}$. Assim, $p(0|1/2)=0$ e $p(1|1/2)=1$.
    \item Se $y=\pi$: $p(x|\pi) = \frac{p(x,\pi)}{1/3}$. Assim, $p(0|\pi)=0$ e $p(1|\pi)=1$.
\end{itemize}

\paragraph{Condicional de $Y$ dado $X=x$:}
\begin{itemize}
    \item Se $x=0$: $p(y|0) = \frac{p(0,y)}{1/3}$. Assim, $p(-1|0)=1$ e $p(y|0)=0$ para $y \neq -1$.
    \item Se $x=1$: $p(y|1) = \frac{p(1,y)}{2/3}$. Assim, $p(1/2|1)=1/2$ e $p(\pi|1)=1/2$.
\end{itemize}

\subsubsection*{3. Obtenção das Funções de Distribuição Acumulada Condicionais}

\paragraph{FDA Condicional $F(x|y) = P(X \le x | Y=y)$:}

A partir das probabilidades condicionais, construímos a FDA para cada
valor de \(y\).

\begin{itemize}
    \item Para $y=-1$: A massa de probabilidade está toda em $X=0$.
    $$ F(x|-1) = 
    \begin{cases}
        0, & \text{se } x < 0 \\
        1, & \text{se } x \ge 0
    \end{cases}
    $$
    \item Para $y=1/2$: A massa de probabilidade está toda em $X=1$.
    $$ F(x|1/2) = 
    \begin{cases}
        0, & \text{se } x < 1 \\
        1, & \text{se } x \ge 1
    \end{cases}
    $$
    \item Para $y=\pi$: A massa de probabilidade está toda em $X=1$.
    $$ F(x|\pi) = 
    \begin{cases}
        0, & \text{se } x < 1 \\
        1, & \text{se } x \ge 1
    \end{cases}
    $$
\end{itemize}

\paragraph{FDA Condicional $F(y|x) = P(Y \le y | X=x)$:}

Analogamente, para cada valor de \(x\).

\begin{itemize}
    \item Para $x=0$: A massa de probabilidade está toda em $Y=-1$.
    $$ F(y|0) = 
    \begin{cases}
        0, & \text{se } y < -1 \\
        1, & \text{se } y \ge -1
    \end{cases}
    $$
    \item Para $x=1$: A massa de probabilidade está distribuída entre $Y=1/2$ e $Y=\pi$.
    $$ F(y|1) = 
    \begin{cases}
        0, & \text{se } y < 1/2 \\
        P(Y \le y | X=1) = p(1/2|1) = 1/2, & \text{se } 1/2 \le y < \pi \\
        P(Y \le y | X=1) = p(1/2|1) + p(\pi|1) = 1, & \text{se } y \ge \pi
    \end{cases}
    $$
\end{itemize}

\newpage

\begin{problema}{}{pr-3}
Suponha que a distribuição conjunta das variáveis aleatórias discretas $X$ e $Y$ está dada por
\begin{center}
    \begin{tabular}{|c|c|c|c|c|} \hline
        $X \diagdown Y$ & 1 & 2 & 3 & 4 \\ \hline
        0 & 0,1 & 0 & 0 & 0 \\ \hline
        -1 & 0,1 & 0,1 & 0 & 0 \\ \hline
        -2 & 0,1 & 0,1 & 0,1 & 0 \\ \hline
        -3 & 0,1 & 0,1 & 0,1 & 0,1 \\ \hline
    \end{tabular}
\end{center}
Calcule:
\begin{enumerate}
    \item $ P(X \geq -1, Y \geq 1) $
    \item As distribuições marginais de $X$ e $Y$ e determine se $X$ e $Y$ são independentes.
    \item Encontre a função de distribuição condicional de $X$ dado $Y.$
\end{enumerate}
\end{problema}

\subsection*{Resolução do Problema 2}

Seja \(p(x,y)\) a função de probabilidade conjunta dada na tabela.

\subsubsection*{Item (a): Cálculo de $ P(X \geq -1, Y \geq 1) $}

O evento \(\{X \ge -1, Y \ge 1\}\) compreende os pares \((x,y)\) tais
que \(x \in \{0, -1\}\) e \(y \in \{1, 2, 3, 4\}\). A probabilidade é a
soma das probabilidades conjuntas para esses pares. \begin{align*}
    P(X \ge -1, Y \ge 1) &= \sum_{x \in \{0,-1\}} \sum_{y=1}^{4} p(x,y) \\
    &= p(0,1) + p(0,2) + p(0,3) + p(0,4) \\
    &\quad + p(-1,1) + p(-1,2) + p(-1,3) + p(-1,4) \\
    &= (0,1 + 0 + 0 + 0) + (0,1 + 0,1 + 0 + 0) \\
    &= 0,1 + 0,2 = 0,3
\end{align*} Portanto, \(\mathbf{P(X \geq -1, Y \geq 1) = 0,3}\).

\subsubsection*{Item (b): Distribuições Marginais e Independência}

As distribuições marginais, \(p_X(x)\) e \(p_Y(y)\), são obtidas somando
as probabilidades ao longo das linhas e colunas da tabela conjunta,
respectivamente.

\begin{center}
    \begin{tabular}{c|cccc|c}
        \toprule
        \diagbox[height=1.2cm]{$X$}{$Y$} & 1 & 2 & 3 & 4 & $p_X(x)$ \\
        \midrule
        0 & 0,1 & 0 & 0 & 0 & 0,1 \\
        -1 & 0,1 & 0,1 & 0 & 0 & 0,2 \\
        -2 & 0,1 & 0,1 & 0,1 & 0 & 0,3 \\
        -3 & 0,1 & 0,1 & 0,1 & 0,1 & 0,4 \\
        \midrule
        $p_Y(y)$ & 0,4 & 0,3 & 0,2 & 0,1 & 1,0 \\
        \bottomrule
    \end{tabular}
\end{center}

\paragraph{Distribuição Marginal de X:}

\(p_X(0) = 0,1\); \quad \(p_X(-1) = 0,2\); \quad \(p_X(-2) = 0,3\);
\quad \(p_X(-3) = 0,4\).

\paragraph{Distribuição Marginal de Y:}

\(p_Y(1) = 0,4\); \quad \(p_Y(2) = 0,3\); \quad \(p_Y(3) = 0,2\);
\quad \(p_Y(4) = 0,1\).

\paragraph{Verificação de Independência:}

Duas variáveis aleatórias \(X\) e \(Y\) são independentes se, e somente
se, \(p(x,y) = p_X(x)p_Y(y)\) para \textbf{todos} os pares \((x,y)\). É
suficiente encontrar um contra-exemplo. Consideremos o par
\((x,y) = (0,1)\):

\begin{itemize}
    \item Da tabela, $p(0,1) = 0,1$.
    \item O produto das marginais é $p_X(0) \cdot p_Y(1) = (0,1) \times (0,4) = 0,04$.
\end{itemize}

Como \(p(0,1) = 0,1 \neq 0,04 = p_X(0)p_Y(1)\), concluímos que as
variáveis aleatórias \textbf{$X$ e $Y$ não são independentes}.

\subsubsection*{Item (c): Função de Distribuição Condicional de X dado Y}

A FDA condicional \(F(x|y) = P(X \le x | Y=y)\) é construída a partir da
PMF condicional \(p(x|y) = p(x,y)/p_Y(y)\).

\paragraph{Caso 1: Y = 1 ($p_Y(1)=0,4$)}

\(p(0|1) = \frac{0,1}{0,4} = \frac{1}{4}\);
\(p(-1|1) = \frac{0,1}{0,4} = \frac{1}{4}\);
\(p(-2|1) = \frac{0,1}{0,4} = \frac{1}{4}\);
\(p(-3|1) = \frac{0,1}{0,4} = \frac{1}{4}\). \(F(x|1) = \begin{cases}
0, & x < -3 \\ 1/4, & -3 \le x < -2 \\ 1/4+1/4 = 1/2, & -2 \le x < -1 \\ 1/2+1/4 = 3/4, & -1 \le x < 0 \\ 1, & x \ge 0
\end{cases}\)

\paragraph{Caso 2: Y = 2 ($p_Y(2)=0,3$)}

\(p(-1|2)=\frac{0,1}{0,3}=\frac{1}{3}\);
\(p(-2|2)=\frac{0,1}{0,3}=\frac{1}{3}\);
\(p(-3|2)=\frac{0,1}{0,3}=\frac{1}{3}\). \(F(x|2) = \begin{cases}
0, & x < -3 \\ 1/3, & -3 \le x < -2 \\ 1/3+1/3=2/3, & -2 \le x < -1 \\ 1, & x \ge -1
\end{cases}\)

\paragraph{Caso 3: Y = 3 ($p_Y(3)=0,2$)}

\(p(-2|3)=\frac{0,1}{0,2}=\frac{1}{2}\);
\(p(-3|3)=\frac{0,1}{0,2}=\frac{1}{2}\). \(F(x|3) = \begin{cases}
0, & x < -3 \\ 1/2, & -3 \le x < -2 \\ 1, & x \ge -2
\end{cases}\)

\paragraph{Caso 4: Y = 4 ($p_Y(4)=0,1$)}

\(p(-3|4)=\frac{0,1}{0,1}=1\). \(F(x|4) = \begin{cases}
0, & x < -3 \\
1, & x \ge -3
\end{cases}\)

\newpage

\begin{problema}{}{pr-2}
    Considere um par de variáveis aleatórias discretas $(X, Y)$ cuja função de distribuição de probabilidade conjunta é $F$, i.e., $F(x,y)=P(X\leq x, Y\leq y),$ $x,y \in \mathbb{R}.$ Sejam $F_X$ e $F_Y$ as funções de distribuição das variáveis aleatórias $X$ e $Y,$ respectivamente (distribuições marginais). Mostre que: $$P(X>x, Y>y)= 1-F_X(x)-F_Y(y)+F(x,y).$$   
\end{problema}

\subsection*{Resolução do Problema 3}

Desejamos provar a identidade \(P(X>x, Y>y)= 1-F_X(x)-F_Y(y)+F(x,y)\).
Esta identidade é frequentemente chamada de função de sobrevivência
conjunta. A prova se baseia na teoria de conjuntos e nos axiomas de
probabilidade de Kolmogorov.

Sejam \(A\) e \(B\) os seguintes eventos:

\begin{itemize}
    \item $A = \{ \omega \in \Omega : X(\omega) > x \}$
    \item $B = \{ \omega \in \Omega : Y(\omega) > y \}$
\end{itemize}

O nosso objetivo é calcular \(P(A \cap B)\).

Consideremos os eventos complementares, \(A^c\) e \(B^c\):

\begin{itemize}
    \item $A^c = \{ \omega \in \Omega : X(\omega) \le x \}$. Por definição da FDA marginal, $P(A^c) = P(X \le x) = F_X(x)$.
    \item $B^c = \{ \omega \in \Omega : Y(\omega) \le y \}$. Por definição da FDA marginal, $P(B^c) = P(Y \le y) = F_Y(y)$.
\end{itemize}

Pelo axioma da probabilidade do complemento, a probabilidade de um
evento \(E\) é \(P(E) = 1 - P(E^c)\). Aplicando isso ao evento
\(A \cap B\): \[ P(A \cap B) = 1 - P((A \cap B)^c) \] Utilizando as Leis
de De Morgan para conjuntos, sabemos que
\((A \cap B)^c = A^c \cup B^c\). Substituindo na equação:
\[ P(A \cap B) = 1 - P(A^c \cup B^c) \] Agora, aplicamos o Princípio da
Inclusão-Exclusão para a probabilidade da união de dois eventos:
\[ P(A^c \cup B^c) = P(A^c) + P(B^c) - P(A^c \cap B^c) \] Vamos
identificar cada termo desta expressão:

\begin{itemize}
    \item $P(A^c) = F_X(x)$.
    \item $P(B^c) = F_Y(y)$.
    \item $A^c \cap B^c = \{X \le x \text{ e } Y \le y\}$. A probabilidade deste evento é, por definição da FDA conjunta, $P(A^c \cap B^c) = P(X \le x, Y \le y) = F(x,y)$.
\end{itemize}

Substituindo estes termos de volta na fórmula da união:
\[ P(A^c \cup B^c) = F_X(x) + F_Y(y) - F(x,y) \] Finalmente, inserimos
este resultado na expressão para \(P(A \cap B)\):
\[ P(A \cap B) = 1 - \left( F_X(x) + F_Y(y) - F(x,y) \right) \]
Distribuindo o sinal negativo, obtemos a identidade desejada:
\[ P(X>x, Y>y) = 1 - F_X(x) - F_Y(y) + F(x,y) \] o que completa a
demonstração. É importante notar que esta prova é geral e se aplica
tanto a variáveis aleatórias discretas quanto contínuas.

\newpage
\begin{problema}{}{pr5}
Sejam $X$ e $Y$ variáveis aleatórias com função de densidade de probabilidade conjunta dada por 
$$ f(x,y) = 
\begin{cases}
    \frac{1}{4}, & \ \text{se} \  \  -1<x<1,  -1<y<1, \\
    0, &  \text{caso contrário.}
\end{cases}
$$

\begin{enumerate}
    \item  Obtenha $P(X+Y> 0)$ e $P(X>0).$
    \item Sejam $Z=X+Y$ e $W=X-Y$ funções lineares das varáveis aleatórias $X$ e $Y.$ Usando o método do Jacobiano obtenha a função de densidade conjunta de $Z$ e $W.$
    \item Obtenha a função de densidade (marginal) de $W$.
    \item  Obtenha a função de densidade condicional de $Z$ dado $W,$ i.e., $f_{Z|W}(z|w).$
\end{enumerate}
\end{problema}

\subsection*{Resolução do Problema 4}

A função de densidade \(f(x,y)\) descreve uma distribuição uniforme
sobre o quadrado \(S = [-1,1] \times [-1,1]\). A área desta região de
suporte é \(A_S = 2 \times 2 = 4\).

\subsubsection*{Item (a): Cálculo de Probabilidades}

Como a distribuição é uniforme, a probabilidade de um evento
\(A \subseteq S\) é dada por \(P(A) = \text{Área}(A) / \text{Área}(S)\).

\paragraph{Cálculo de $P(X+Y>0)$:}

O evento corresponde à região \(R_1 = \{ (x,y) \in S : x+y > 0 \}\), ou
\(y > -x\). A linha \(y=-x\) divide o quadrado \(S\) em duas metades de
área igual. A região \(y > -x\) é a metade superior do quadrado.
Portanto, a Área(\(R_1\)) =
\(\frac{1}{2} \text{Área}(S) = \frac{1}{2} \times 4 = 2\). A
probabilidade é: \$ P(X+Y\textgreater0) =
%\frac{\text{Área}(R_1)}{\text{Área}(S)} = \frac{2}{4} = \frac{1}{2}\$.

\paragraph{Cálculo de $P(X>0)$:}

O evento corresponde à região \(R_2 = \{ (x,y) \in S : x > 0 \}\). Esta
região é o retângulo \([0,1] \times [-1,1]\), que é a metade direita do
quadrado \(S\). A área é
\(\text{Área}(R_2) = (1-0) \times (1-(-1)) = 1 \times 2 = 2\). A
probabilidade é: \$ P(X\textgreater0) =
%\frac{\text{Área}(R_2)}{\text{Área}(S)} = \frac{2}{4} = \frac{1}{2}\$.

\subsubsection*{Item (b): Método do Jacobiano}
\paragraph{1. Transformação Inversa:}

Dada a transformação \(Z=X+Y\) e \(W=X-Y\), resolvemos para \(X\) e
\(Y\):

\begin{itemize}
    \item $Z+W = (X+Y)+(X-Y) = 2X \implies X = \frac{Z+W}{2}$
    \item $Z-W = (X+Y)-(X-Y) = 2Y \implies Y = \frac{Z-W}{2}$
\end{itemize}

\paragraph{2. Jacobiano da Transformação Inversa:}

O Jacobiano \(J\) é o determinante da matriz de derivadas parciais de
\((x,y)\) em relação a \((z,w)\):
\[ J = \det \begin{pmatrix} \frac{\partial x}{\partial z} & \frac{\partial x}{\partial w} \\ \frac{\partial y}{\partial z} & \frac{\partial y}{\partial w} \end{pmatrix} = \det \begin{pmatrix} 1/2 & 1/2 \\ 1/2 & -1/2 \end{pmatrix} = \left(\frac{1}{2}\right)\left(-\frac{1}{2}\right) - \left(\frac{1}{2}\right)\left(\frac{1}{2}\right) = -\frac{1}{4} - \frac{1}{4} = -\frac{1}{2} \]
O valor absoluto do Jacobiano é \(|J| = 1/2\).

\paragraph{3. Nova Região de Suporte:}

O suporte original é \(-1<x<1\) e \(-1<y<1\). Substituímos \(X\) e
\(Y\):

\begin{itemize}
    \item $-1 < \frac{z+w}{2} < 1 \implies -2 < z+w < 2$
    \item $-1 < \frac{z-w}{2} < 1 \implies -2 < z-w < 2$
\end{itemize}

Esta região \(S'\) no plano \((z,w)\) é um quadrado rotacionado com
vértices em \((2,0), (0,2), (-2,0), (0,-2)\).

\paragraph{4. Densidade Conjunta de $(Z, W)$:}

A densidade é \(f_{Z,W}(z,w) = f_{X,Y}(x(z,w), y(z,w)) \cdot |J|\).
\[ f_{Z,W}(z,w) = \frac{1}{4} \cdot \frac{1}{2} = \frac{1}{8} \] Assim,
a FDP conjunta de \((Z,W)\) é: \[ f_{Z,W}(z,w) = 
\begin{cases}
    \frac{1}{8}, & \text{se } (z,w) \in S' \\
    0, &  \text{caso contrário.}
\end{cases}
\]

\subsubsection*{Item (c): Densidade Marginal de $W$}

Para obter \(f_W(w)\), integramos \(f_{Z,W}(z,w)\) sobre \(z\). Para um
\(w \in (-2,2)\) fixo, os limites de \(z\) são dados por
\(-2 < z+w < 2 \implies -2-w < z < 2-w\) e
\(-2 < z-w < 2 \implies w-2 < z < w+2\). Combinando:
\(\max(-2-w, w-2) < z < \min(2-w, w+2)\). Isto simplifica para
\(-2+|w| < z < 2-|w|\). \begin{align*}
    f_W(w) = \int_{-\infty}^{\infty} f_{Z,W}(z,w) \,dz &= \int_{-2+|w|}^{2-|w|} \frac{1}{8} \,dz \\
    &= \frac{1}{8} [z]_{-2+|w|}^{2-|w|} = \frac{1}{8} \left( (2-|w|) - (-2+|w|) \right) \\
    &= \frac{1}{8} (4 - 2|w|) = \frac{2-|w|}{4}
\end{align*} A densidade marginal de \(W\) (uma distribuição triangular)
é: \[ f_W(w) = 
\begin{cases}
    \frac{2-|w|}{4}, & \text{se } -2 < w < 2 \\
    0, &  \text{caso contrário.}
\end{cases}
\]

\subsubsection*{Item (d): Densidade Condicional $f_{Z|W}(z|w)$}

A densidade condicional é
\(f_{Z|W}(z|w) = \frac{f_{Z,W}(z,w)}{f_W(w)}\), para \(f_W(w)>0\). \[
f_{Z|W}(z|w) = \frac{1/8}{(2-|w|)/4} = \frac{4}{8(2-|w|)} = \frac{1}{2(2-|w|)}
\] O suporte de \(z\) dado \(w\) é \(-2+|w| < z < 2-|w|\). \[
f_{Z|W}(z|w) = 
\begin{cases}
    \frac{1}{2(2-|w|)}, & \text{para } -2+|w| < z < 2-|w| \text{ e } w \in (-2,2) \\
    0, &  \text{caso contrário.}
\end{cases}
\] Isto indica que, dado \(W=w\), \(Z\) segue uma distribuição uniforme
no intervalo \((-2+|w|, 2-|w|)\).

\newpage
\begin{problema}{}{pr-4}
    Considere  a {\it convolução} $f_X * f_Y$ entre as funções de densidade das variáveis aleatórias  $X$ e $Y,$ i.e., a função de densidade da variável aleatória $Z=X+Y.$ Mostre que o operador de convolução $ ( * )$ é: 
    \begin{enumerate}
        \item comutativo: $f_X * f_Y = f_Y * f_X$
        \item distributivo: $f_Z*(f_X+f_Y ) = f_Z*f_X+f_Z*f_Y$
        \item  associativo: $(f_Z * f_X) * f_Y = f_Z *(f_X * f_Y )$
    \end{enumerate}
\end{problema}

\subsection*{Resolução do Problema 5}

A convolução de duas funções integráveis \(g\) e \(h\) é definida como
\((g * h)(t) = \int_{-\infty}^{\infty} g(x) h(t-x) \,dx\). Se \(X\) e
\(Y\) são V.A.s independentes com FDPs \(f_X\) e \(f_Y\), a FDP da soma
\(S=X+Y\) é \(f_S = f_X * f_Y\).

\subsubsection*{Item (a): Comutatividade}

Desejamos provar que \((f_X * f_Y)(t) = (f_Y * f_X)(t)\).
\[ (f_X * f_Y)(t) = \int_{-\infty}^{\infty} f_X(x) f_Y(t-x) \,dx \]
Realizamos a mudança de variável \(u = t-x\). Assim, \(x = t-u\) e
\(dx = -du\). Os limites de integração se invertem: quando
\(x \to \infty\), \(u \to -\infty\) e quando \(x \to -\infty\),
\(u \to \infty\). \begin{align*}
    (f_X * f_Y)(t) &= \int_{\infty}^{-\infty} f_X(t-u) f_Y(u) (-du) \\
    &= \int_{-\infty}^{\infty} f_Y(u) f_X(t-u) \,du \quad (\text{invertendo os limites e o sinal}) \\
    &= (f_Y * f_X)(t)
\end{align*} Portanto, o operador de convolução é comutativo.

\subsubsection*{Item (b): Distributividade}

Desejamos provar que \(f_Z * (f_X + f_Y) = (f_Z * f_X) + (f_Z * f_Y)\).
\begin{align*}
    (f_Z * (f_X+f_Y))(t) &= \int_{-\infty}^{\infty} f_Z(x) (f_X+f_Y)(t-x) \,dx \\
    &= \int_{-\infty}^{\infty} f_Z(x) [f_X(t-x) + f_Y(t-x)] \,dx \quad (\text{soma de funções}) \\
    &= \int_{-\infty}^{\infty} [f_Z(x)f_X(t-x) + f_Z(x)f_Y(t-x)] \,dx \quad (\text{distributividade do produto}) \\
    &= \int_{-\infty}^{\infty} f_Z(x)f_X(t-x) \,dx + \int_{-\infty}^{\infty} f_Z(x)f_Y(t-x) \,dx \quad (\text{linearidade da integral}) \\
    &= (f_Z * f_X)(t) + (f_Z * f_Y)(t)
\end{align*} Portanto, o operador é distributivo sobre a adição.

\subsubsection*{Item (c): Associatividade}

Desejamos provar que
\(((f_Z * f_X) * f_Y)(t) = (f_Z * (f_X * f_Y))(t)\). Começamos pelo lado
esquerdo. Seja \(g = f_Z * f_X\), onde
\(g(y) = \int_{-\infty}^{\infty} f_Z(x) f_X(y-x) \,dx\). \begin{align*}
    ((f_Z * f_X) * f_Y)(t) &= (g * f_Y)(t) = \int_{-\infty}^{\infty} g(y) f_Y(t-y) \,dy \\
    &= \int_{-\infty}^{\infty} \left( \int_{-\infty}^{\infty} f_Z(x) f_X(y-x) \,dx \right) f_Y(t-y) \,dy
\end{align*} Pelo Teorema de Fubini-Tonelli, podemos trocar a ordem de
integração:
\[ = \int_{-\infty}^{\infty} f_Z(x) \left( \int_{-\infty}^{\infty} f_X(y-x) f_Y(t-y) \,dy \right) \,dx \]
Analisamos a integral interna. Façamos a substituição \(u = y-x\), o que
implica \(y = u+x\) e \(dy = du\).
\[ \int_{-\infty}^{\infty} f_X(u) f_Y(t-(u+x)) \,du = \int_{-\infty}^{\infty} f_X(u) f_Y((t-x)-u) \,du \]
Esta integral é, por definição, \((f_X * f_Y)(t-x)\). Substituindo de
volta: \[ \int_{-\infty}^{\infty} f_Z(x) (f_X * f_Y)(t-x) \,dx \] Esta
expressão é a definição de \((f_Z * (f_X * f_Y))(t)\). Portanto, a
associatividade é válida.

\end{document}
