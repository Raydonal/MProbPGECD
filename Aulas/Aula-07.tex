%\def\R{$\textsf{R}$}
%\def\S{$\textsf{S}$}

%\renewcommand{\labelitemii}{$\circ$}
\newcommand{\sfa}{a}
\newcommand{\worst}{\mbox{\em worst}}
\newcommand{\best}{\mbox{\em best}}
\newcommand{\regret}{\mbox{\em regret}}
\newcommand{\opt}{\mbox{\em opt}}
\newcommand{\join}{\bowtie}
\newcommand{\lE}{\underline{E}}
\newcommand{\uE}{\overline{E}}
\newcommand{\heads}{{\it heads}}
\newcommand{\tails}{{\it tails}}

\newcommand{\A}{{\cal A}}
\newcommand{\B}{{\cal B}}
\newcommand{\C}{{\cal C}}
\newcommand{\D}{{\cal D}}
\newcommand{\E}{{\cal E}}
\newcommand{\F}{{\cal F}}
\newcommand{\G}{{\cal G}}
%\newcommand{\H}{{\cal H}}
\newcommand{\I}{{\cal I}}
\newcommand{\J}{{\cal J}}
\newcommand{\K}{{\cal K}}
%\newcommand{\L}{{\cal L}}
\newcommand{\M}{{\cal M}}
\newcommand{\N}{{\cal N}}
%\newcommand{\O}{{\cal O}}
\newcommand{\Ocal}{{\cal O}}
\newcommand{\Hcal}{{\cal H}}
\renewcommand{\P}{{\cal P}}
\newcommand{\Q}{{\cal Q}}
\newcommand{\R}{{\cal R}}
%\newcommand{\S}{{\cal S}}
\newcommand{\T}{{\cal T}}
\newcommand{\U}{{\cal U}}
\newcommand{\V}{{\cal V}}
\newcommand{\W}{{\cal W}}
\newcommand{\X}{{\cal X}}
\newcommand{\Y}{{\cal Y}}
\newcommand{\Z}{{\cal Z}}


\newcommand{\IR}{\mathbb{R}}
\newcommand{\dfn}{\begin{definition}}
\newcommand{\edfn}{\end{definition}}
\newcommand{\thm}{\begin{theorem}}
\newcommand{\ethm}{\end{theorem}}
\newcommand{\xam}{\begin{example}}
\newcommand{\exam}{\end{example}}
\newcommand{\inter}{\cap}
\newcommand{\union}{\cup}




\documentclass[t, 8pt, seriff]{beamer}


%\documentclass[a4paper,xcolor=svgnames]{beamer} 
\usepackage[portuguese]{babel}
\usepackage[utf8]{inputenc}
\usepackage{times}
\usepackage{amsmath,amsthm}
\usepackage{amssymb,latexsym}
\usepackage{graphics}
%\usepackage{graphicx}

\usepackage{multimedia}
% \usepackage{movie15}
\usepackage{media9}

\usetheme{default}
%\usetheme{Singapore}
%\usetheme{PaloAlto} 
\usetheme{Boadilla}
% other themes: AnnArbor, Antibes, Bergen, Berkeley, Berlin, Boadilla, boxes, CambridgeUS, Copenhagen, Darmstadt, default, Dresden, Frankfurt, Goettingen,
% Hannover, Ilmenau, JuanLesPins, Luebeck, Madrid, Maloe, Marburg, Montpellier, PaloAlto, Pittsburg, Rochester, Singapore, Szeged, boxes, default

\useoutertheme{infolines}
%\usefonttheme{serif}
% you can also specify font themes: default, professionalfonts, serif, structurebold, structureitalicserif, structuresmallcapsserif

%\definecolor{vermelho}{RGB}{100,30,40}
%\definecolor{vermelholys}{RGB}{132,158,139}
%\definecolor{vermelholyslys}{RGB}{173,190,177}
%\definecolor{vermelholyslyslys}{RGB}{214,223,216}


%\usecolortheme[named=vermelho]{structure}




 



%\documentclass[a4paper,xcolor=svgnames]{beamer} 
%\usepackage[brazil]{babel}
%\usepackage[latin1]{inputenc}
\usepackage{ragged2e}
\usepackage{bm}
\usepackage[T1]{fontenc}
%\usepackage{amsmath,amsthm,amsfonts,amssymb} 
\usepackage{multirow}
%\usetheme{CambridgeUS} 
%\setbeamercolor{normal text}{bg=white}
\usepackage {graphicx,color}

\usepackage{wrapfig} % inserir a figura ao lado do texto
\usepackage[dvips]{epsfig} % inserir figuras de extensao post script (ps)
\usepackage{textcomp}
% \usepackage{undertilde} % colocar o til abaixo do x
\usepackage{multicol} % cor na linha
\usepackage{tabularx}
\usepackage{rotating} %rotacionar figuras e tabelas


\usepackage{ragged2e}
%\justifying


\usepackage{tikz}
\usetikzlibrary{trees}


\newtheorem{lema}{Lema}
\newtheorem{defi}{Definição}
\newtheorem{teo}{Teorema}
\newtheorem{corol}{Corolário}
\newtheorem{prop}{Proposição}


\newtheoremstyle{Exercício}{}{}{\rm}{}{\bf $\bigstar$ }{:}{ }{} %% \scshape para mudar
\theoremstyle{Exercício}
\newtheorem{exer}{Exercício}

\theoremstyle{plain}
\newtheoremstyle{Exemplo}{}{}{\rm}{}{\bf $\rhd$ }{:}{ }{} %% \scshape para mudar
%o tamanho a maiusculo
\theoremstyle{Exemplo}
\newtheorem{exem}{Exemplo}

% 
% \theoremstyle{plain}
% \newtheoremstyle{Nota}{}{}{\rm}{}{\bf\scshape}{:}{ }{}
% \theoremstyle{Nota}
 \newtheorem{nota}{Nota}






%\setlength{\rightskip}{0pt}
%\setlength{\leftskip}{0pt}
%\setlength{\spaceskip}{0pt}
%\setlength{\xspaceskip}{0pt}



\newcommand{\fullpage}[1]{
\begin{frame}
 #1
\end{frame}
}


\setbeamersize{text margin left=3em, text margin right=3em}



\setbeamertemplate{theorems}[numbered]



\definecolor{links}{HTML}{2A1B81}
\hypersetup{colorlinks,linkcolor=,urlcolor=links}


\graphicspath{{./graphics/}} 			% path das figuras (recomendável)

\newcommand{\cor}[1]{ \{{#1}\}}


\title[Probabilidade]{  Probabilidade (PGE950) }
\author[ Raydonal Ospina 
%\textcopyright 
\ ]{
	%Probabilidade\\ 
	% Sessão 1 \\
	%Probabilidade\\ 
	Sessão 7 \\
	${}$ \\
	Raydonal Ospina  }
\date[PGE950 - \today ]{{\tiny PGE950 }}

\institute[UFPE]{Departamento de Estatística\\
	Universidade Federal de Pernambuco\\
	Recife/PE}

\usecolortheme[rgb={0,0.3,0.5}]{structure}


%%%%%%%%%%%%%%%%%%%%%%%%%%%%%%%%%%%%%%%%%%%%%%%%%%%%%%%%%%%%%%%%%%%%%%
\begin{document}
% \SweaveOpts{concordance=TRUE}
\begin{frame}
  \titlepage
\end{frame}


%%%%%%%%%%%%%%%%%%%%%%%%%%%%%%%%%%%%%%%%%%%%%%%%%%%%%%%%%%%%%%%%%%%%%%%%%%%%%%%%%%%%%%%%%%%%%%%%%%%%%%%%%%

\section{Esperança}
\begin{frame}
\frametitle{Esperança}
	\begin{block}{Interpretação}
	\begin{enumerate}
		\item Parâmetro $\mu$ de uma medida de probabilidade, função de distribuição, ou função probabilidade de massa, também conhecido como média.
		\item Um operador linear em um conjunto de variáveis aleatórias que
		retorna um valor típico da variável aleatória interpretado como uma
		medida de localização da variável aleatória.
		\item média do resultado de repetidos experimentos independentes no longo prazo.
		\item preço justo de um jogo com pagamentos descritos por $X$.
	\end{enumerate}
\end{block}
\end{frame}

\begin{frame}
\begin{block}{Esperança - Caso Discreto}
Considere que um dado é lançado 1000 vezes. Uma maneira de
calcular este resultado médio seria somar todos os resultados e
dividir por 1000. Uma maneira alternativa seria calcular a fração
$p(k)$ de todos os lançamentos que tiveram resultado igual a $k$ e
calcular o resultado médio através da soma ponderada:
$$1p(1)+2p(2)+3p(3)+4p(4)+5p(5)+6p(6).$$
Quando o número de lançamentos se torna grande as frações de
ocorrência dos resultados tendem a probabilidade de cada resultado.
\end{block}

\begin{defi}
	Se $X$ é uma variável aleatória discreta assumindo valores
	$\{x_1,x_2,x_3,\ldots\}$ com probabilidade $\{p_1,p_2,p_3,\ldots\}$,
	respectivamente, então sua esperança é dada pela fórmula
	$$EX=\sum_{i:x_i<0}x_ip_i+\sum_{i:x_i\geq 0}x_ip_i,$$
	desde que pelo menos um dos somatórios seja finito. Em caso os dois
	somatórios não sejam finitos, a esperança não existe.
\end{defi}
\end{frame}

\begin{frame}
\frametitle{Exemplos}
\begin{exem}[Aleatória]
Se $X\in\{1,2,\ldots,n\}$ for uma variável aleatória
com distribuição de probabilidade aleatória com parâmetro $n$, temos
que sua esperança é dada por:
\begin{eqnarray}
& & EX=\sum_{k=1}^{n}kp(k)=\sum_{k}^n k{\frac{1}{n}} \nonumber \\
& & =\frac{1}{n}\sum_{k}^n k=\frac{1}{n}\frac{n(n+1)}{2}=\frac{n+1}{2}.\nonumber
\end{eqnarray}
Onde utilizamos a fórmula da soma dos primeiros $n$ termos de uma
progressão aritmética.
\end{exem}

\begin{exem}{Bernoulli}

Se $X\in\{0,1\}$ for uma variável aleatória
com distribuição de probabilidade Bernoulli com parâmetro $p$, temos
que sua esperança é dada por:
$$EX=0(1-p)+1(p)=p.$$

\end{exem}
\end{frame}

\begin{frame}
%\frametitle{\textbf{Exemplos}}
%\baselineskip=13pt
\begin{exem}[Binomial]
%
Se $X$ for uma variável aleatória
com distribuição de probabilidade Binomial com parâmetros $n$ e $p$,
temos que sua esperança é dada por:
\begin{eqnarray}
& & EX=\sum_{k=0}^{n}
k\binom{n}{k}p^k(1-p)^{n-k}= 
 \sum_{k=1}^{n}k\frac{n!}{k!(n-k)!}p^k(1-p)^{n-k}
\nonumber\\
& &
\sum_{k=1}^{n}n\frac{(n-1)!}{(k-1)!(n-k)!}p^k(1-p)^{n-k} =np\sum_{k=1}^{n}\binom{n-1}{k-1}p^{k-1}(1-p)^{n-k}=np.
\nonumber
\end{eqnarray}
Onde utilizamos o Teorema Binomial na última igualdade.
\end{exem}
%\end{frame}
%%
%%\begin{frame}
%%\frametitle{\textbf{Exemplos}}
%%\baselineskip=13pt
\begin{exem}[Geométrica]
%
Se $X$ for uma variável aleatória
com distribuição de probabilidade Geométrica com parâmetro $\beta$,
temos que sua esperança é dada por:
\begin{eqnarray}
& &
EX=\sum_{k=0}^{\infty}k(1-\beta)\beta^{k}=\sum_{k=1}^{\infty}k(1-\beta)\beta^{k}
=\sum_{k=1}^{\infty}\sum_{j=1}^{k}(1-\beta)\beta^{k}
=(1-\beta)\sum_{j=1}^{\infty}\sum_{k=j}^{\infty}\beta^{k}\nonumber \\
& & =\sum_{j=1}^{\infty}\beta^j=\frac{\beta}{1-\beta}
\nonumber
\end{eqnarray}
Onde utilizamos a fórmula da soma infinita de uma progressão
geométrica com razão $\beta$.
\end{exem}
\end{frame}
%
\begin{frame}
%\frametitle{\textbf{Exemplos}}
%\baselineskip=13pt
\begin{exem}[Binomial Negativa]
Se $X$ for uma variável aleatória
com distribuição de probabilidade Binomial Negativa com parâmetros
$r$ e $p$, temos que sua esperança é dada por:
\begin{eqnarray}
& &
EX=\sum_{k=r-1}^{\infty}k\binom{k}{r-1}p^{r}(1-p)^{k-r+1}
\nonumber \\
& & =\Big(\sum_{k=r-1}^{\infty}(k+1)\binom{k}{r-1}p^{r}(1-p)^{k-r+1}\Big)-1
\nonumber\\
& &
=\Big(\sum_{k=r-1}^{\infty}\frac{(k+1)k!}{(r-1)!(k-r+1)!}p^r(1-p)^{k-r+1}\Big)-1\nonumber
\\
& &
=\frac{r}{p}\Big(\sum_{k=r-1}^{\infty}\frac{(k+1)!}{r!(k+1-r)!}p^{r+1}(1-p)^{k+1-r}\Big)-1\nonumber
\end{eqnarray}

Substituindo $j=k+1$ e $s=r+1$ no somatório, temos
\begin{eqnarray}
& & EX=
\frac{r}{p}\Big(\sum_{j=s-1}^{\infty}\frac{(j)!}{(s-1)!(j-s+1)!}p^{s}(1-p)^{j-s+1}\Big)-1\nonumber=\frac{r}{p}-1\nonumber
\end{eqnarray}
Onde utilizamos o fato que o somatório é igual soma da função
probabilidade de massa de uma variável aleatória Binomial Negativa
para todos os valores que tem probabilidade positiva, e portanto, é
igual a 1.


\end{exem}
\end{frame}
%
\begin{frame}
%\frametitle{\textbf{Exemplos}}
%\baselineskip=13pt
\begin{exem}[Poisson]
%
Se $X$ for uma variável aleatória
com distribuição de probabilidade Poisson com parâmetros $\lambda$,
temos que sua esperança é dada por:
$$EX=\sum_{k=0}^{\infty}k\frac{e^{-\lambda}\lambda^k}{k!}=\sum_{k=1}^{\infty}k\frac{e^{-\lambda}\lambda^k}{k!}=\lambda\sum_{k=1}^{\infty}\frac{e^{-\lambda}\lambda^{k-1}}{(k-1)!}=\lambda.$$

\end{exem}
%\end{frame}
%
%\begin{frame}
%\frametitle{\textbf{Exemplos}}
%\baselineskip=13pt
\begin{exem}[Zeta]
%
Se $X$ for uma variável aleatória
com distribuição de probabilidade Zeta com parâmetro $\alpha>2$,
temos que sua esperança é dada por:
$$EX=\sum_{k=1}^{\infty}k\frac{k^{-\alpha}}{\zeta(\alpha)}=\frac{1}{\zeta(\alpha)}\sum_{k=1}^{\infty}k^{-(\alpha-1)}=\frac{\zeta(\alpha-1)}{\zeta(\alpha)},$$
onde $\zeta(\alpha)=\sum_{k=1}^{\infty}k^{-\alpha}$.

\end{exem}
\end{frame}
%
\begin{frame}
%\frametitle{\textbf{Exemplos}}
%\baselineskip=13pt
\begin{exem}[Hipergeométrica]
	%
Se $X$ for uma variável aleatória
com distribuição de probabilidade Hipergeométrica com parâmetro
$N,D,n$, temos que sua esperança é dada por:

\begin{eqnarray}
& &
EX=\sum_{k=0}^{n}k\frac{\binom{D}{k}\binom{N-D}{n-k}}{\binom{N}{n}}=\sum_{k=1}^{n}\frac{D!(N-D)!(N-n)!n!}{k!(D-k)!(n-k)!(N-D-n+k)!N!}\nonumber\\
& & =
\frac{nD}{N}\sum_{k=1}^{n}\frac{(D-1)!(N-D)!(N-n)!(n-1)!}{(k-1)!(D-k)!(n-k)!(N-D-n+k)!(N-1)!}\nonumber \\ & & =\frac{nD}{N}\sum_{k=1}^{n}\frac{\binom{D-1}{k-1}\binom{N-D}{n-k}}{\binom{N-1}{n-1}}\nonumber
\end{eqnarray}
%
%\end{block}
%\end{frame}
%
%\begin{frame}
%\frametitle{\textbf{Exemplos}}
%\baselineskip=13pt
%\begin{exem}{Hipergeométrica}
%
%
Substituindo no somatório $D^*=D-1,k^*=k-1,n^*=n-1$ e $N^*=N-1$,
temos
$$EX=\frac{nD}{N}\sum_{k^*=0}^{n^*}\frac{\binom{D^*}{k^*}\binom{N^*-D^*}{n^*-k^*}}{\binom{N^*}{n^*}}=\frac{nD}{N}.$$
Onde utilizamos o fato que o somatório é igual soma da função
probabilidade de massa de uma variável aleatória Hipergeométrica
para todos os valores que tem probabilidade positiva, e portanto, é
igual a 1.
%
%
\end{exem}
\end{frame}
%
\begin{frame}
%\frametitle{\textbf{As integrais de Riemman-Stieltjes e de Lebesgue-Stieltjes}}
%\baselineskip=13pt
\begin{block}{As integrais de Riemman-Stieltjes e de Lebesgue-Stieltjes}
%
%
\begin{itemize}
\item Antes de introduzirmos a definição geral da Esperança de uma variável aleatória qualquer, vamos estudar um pouco sobre as integrais de Riemann-Stieltjes e de Lebesgue-Stieltjes. \\ 

%Antes de darmos as definições das integrais de Riemman-Stieltjes e Lebesgue-Stieltjes, 
\item Vamos relembrar a definição da integral de Riemann. Uma partição $P$ do intervalo $[a,b]$ é uma sequência de pontos $\{x_1,\ldots,x_n\}$ tal que $a=x_1<x_2<\cdots<x_{n}=b$; a norma da partição $P$ é definida como sendo $\max_{1\leq i\leq n-1}(x_{i+1}-x_i)$. \\ 

\item Suponha que $\varphi$ seja uma função real qualquer definida no intervalo $[a,b]$. Diz-se que esta função é Riemann integrável se as somas de Riemann
$$\sum_{i=1}^{n-1}\varphi(y_i)(x_{i+1}-x_i),$$
onde $y_i\in[x_i,x_{i+1}]$, convergem quando a norma de $P$ tende a zero e este limite é independente da escolha dos $y_i$'s e da partição $P$. Se esta integral existe denota-se o limite por $\int_{a}^{b}\varphi(x) dx$.

\item A integral de Riemann-Stieltjes é uma generalização de integral de Riemann.
\end{itemize}

\end{block}
\end{frame}
%
\begin{frame}
%\frametitle{\textbf{As integrais de Riemman-Stieltjes e de Lebesgue-Stieltjes}}
%\baselineskip=13pt
\vspace{-0.3cm}
\begin{block}{Integral de Riemman-Stieltjes}
Se $\varphi$ é uma função contínua definida no intervalo $[a,b]$ e $F$ é uma função de distribuição, define-se a integral de Riemann-Stieltjes de $\varphi$ em $[a,b]$, em relação a $F$, como o limite de somas de Riemann da forma
$$\sum_{i=1}^{n-1}\varphi(y_i)[F(x_{i+1})-F(x_i)],$$
em que $a=x_1<x_2<\cdots<x_{n}=b$, $y_i$ é um ponto arbitrário de $[x_i,x_{i+1}]$, e toma-se o limite quando a norma de partição $P$ tende a zero. Tal limite existe e é finito sob as condições descritas, e é representado por
$$\int_{a}^{b}\varphi(x)dF(x).$$
A função $\varphi$ é chamada de integrando e $F$ de integrador. O limite acima existe mesmo que $F$ não seja uma função de distribuição basta que ela seja de variação limitada.
\end{block}
\vspace{-0.2cm}
{\small
\begin{defi}[Variação Total]
Define-se variação total de uma função $f$ em $[a,b]$ pelo funcional:
$$V(f,[a,b])=\sup \Big\{\sum_{i=1}^n|f(x_{i+1})-f(x_i)|\Big\},$$
em 	que o supremo é tomado sobre todas as possíveis partições do intervalo fechado $[a,b]$. Uma função é de variação limitada se $V(f,[a,b])<\infty$.
\end{defi}
}

\end{frame}
%
%\begin{frame}
%\frametitle{\textbf{As integrais de Riemman-Stieltjes e de Lebesgue-Stieltjes}}
%\baselineskip=13pt
%\begin{block}{}
%
%
%\end{block}
%\end{frame}
%
\begin{frame}
%\frametitle{\textbf{As integrais de Riemman-Stieltjes e de Lebesgue-Stieltjes}}
%\baselineskip=13pt
\begin{block}{}
	\begin{itemize}
%
\item A integral de Rieman-Stieltjes sobre a reta é uma integral imprópria definida da mesma maneira que a integral imprópria de Riemann:
$$\int_{-\infty}^{\infty}\varphi(x)dF(x)=\lim_{a\rightarrow-\infty,b\rightarrow\infty}\int_{a}^{b}\varphi(x)dF(x),$$
se o limite existe.

\item Pode-se estender esta definição para outras funções $\varphi$ além das contínuas. 

\item Para uma função qualquer $\varphi$, define-se $\int_{a}^{b}\varphi(x)dF(x)$ como sendo o limite das somas de Riemann descritas acima quando a norma da partição tende a zero, se este limite existe e é independente das escolhas dos $y_i$'s e da partição $P$. 

\item O problema é que mesmo para funções bem simples este limite pode não existir como mostra o próximo exemplo:
\end{itemize}
\end{block}
\begin{exem}
	%
	%%\begin{example}
	Seja $F_0(x)=1$ se $x\geq 0$, e $F_0(x)=0$, caso contrário. Consideremos a integral de Riemann-Stieltjes de $F_0$ em $[-1,1]$ em relação a $F_0$. Note que se zero não é um dos pontos da partição, de modo que $x_i<0<x_{i+1}$ para algum $i$, com $F_0(x_{i+1})-F_0(x_i)=1$, então o valor da soma de Riemann pode ser igual a 1 ou 0 dependendo de $y_i$ ser maior que 0, ou não.
	%%\end{example}
	%
\end{exem}
\end{frame}
%
\begin{frame}
%\frametitle{\textbf{Exemplo}}
%\baselineskip=13pt

%\end{frame}
%
%\begin{frame}
%\frametitle{\textbf{As integrais de Riemman-Stieltjes e de Lebesgue-Stieltjes}}
%\baselineskip=13pt
\begin{block}{}
\begin{itemize}
\item Uma integral mais robusta que não sofre desta deficiência é a integral de Lebesgue-Stieltjes. 

\item A idéia da integral de Lebesgue-Stieltjes é particionar a imagem da função $\varphi$ ao invés de particionar o seu domínio. 

\item Diz-se que uma partição $P'$ é um refinamento de $P$ se $P\subseteq P'$, ou seja, quando os intervalos da partição $P$ são particionados na partição $P'$. 

\item Suponha que $\varphi$ seja {\bf não negativa e mensurável em relação a $\sigma$-álgebra de Borel}. Seja $\lambda$  uma medida nos reais, ou seja, uma função cujo domínio é a $\sigma$-álgebra de Borel que tem como imagem do conjunto vazio o zero, é não-negativa e é $\sigma$-aditiva. 

\item Dada uma sequência $\{P_1,P_2,\ldots\}$ de partições de $[0,\infty)$ onde $P_n=\{y_1,y_2,\ldots,y_n\}$, $y_n\rightarrow\infty$, $P_{i+i}$ é um refinamento de $P_i$, e a norma de $P_n$ tende a zero quando $n\rightarrow\infty$, define-se a soma de Lebesgue em relação a partição $P_n$ como sendo,
$$\sum_{i=1}^{n-1}y_i\lambda(\{x:y_i\leq \varphi(x)<y_{i+1}\})+y_n\lambda(\{x:\varphi(x)\geq y_n\}).$$

\item A integral de Lebesgue-Stieltjes de $\varphi$ em relação a $\lambda$ é definida como sendo igual ao limite das somas de Lebesgue,
quando $n\rightarrow\infty$. Dadas as condições acima este limite sempre existe (pode ser $+\infty$) e é denotado por $\int \varphi d\lambda$.
%
\end{itemize}
\end{block}
\end{frame}
%
\begin{frame}
%\frametitle{\textbf{As integrais de Riemman-Stieltjes e de Lebesgue-Stieltjes}}
%\baselineskip=13pt
\vspace{1cm}
\begin{block}{}
\begin{itemize}
\item Para uma função mensurável $\varphi$ qualquer, podemos escrever $\varphi=\varphi^+-\varphi^-$, onde $\varphi^+=\max(\varphi,0)$, a parte positiva de $\varphi$, e $\varphi^-=-\min(\varphi,0)$, o módulo da parte negativa de $\varphi$, são funções não-negativas e portanto possuem integral de Lebesgue-Stieltjes. 

\item Se $\varphi^+$ ou $\varphi^-$ possui integral de Lebesgue-Stieltjes finita em relação a $\mu$, define-se a integral de Lebesgue-Stieltjes de $\varphi$ em relação a $\lambda$ como sendo
$$\int\varphi d\lambda=\int\varphi^+d\lambda-\int\varphi^-d\lambda.$$

\item Se $\lambda$ for uma medida de probabilidade em $(\IR,\B)$ e $F$ for a distribuição de probabilidade acumulada associada a variável aleatória $X(\omega)=\omega$, então escreve-se $$\int \varphi(x) dF(x)$$ (ou simplesmente, $\int \varphi dF$) para denotar $\int \varphi d\lambda$.
%
\end{itemize}
\end{block}
\end{frame}
%
\begin{frame}
%\frametitle{\textbf{As integrais de Riemman-Stieltjes e de Lebesgue-Stieltjes}}
%\baselineskip=13pt
\vspace{1cm}
\begin{block}{}
\begin{itemize}%
%
\item Em geral, usa-se a notação $\int \varphi(x) dF(x)$ não somente para funções de distribuição, mas para qualquer função $F$ que pode ser escrita como a diferença de duas funções monótonas não-decrescentes, limitadas e contínuas à direita. 

\item Se $G$ for uma função monótona não-decrescente, limitada e contínua à direita, então dado um intervalo qualquer $I=[x_1,x_2]$, defina $\nu(I)=G(x_2)-G(x_1)$, então usa-se a notação $\int \varphi(x) dG(x)$ para denotar a integral $\int \varphi(x) d\nu$, onde $\nu$ é a única medida que satisfaz $\nu(I)=G(x_2)-G(x_1)$ para todo intervalo $I$. 

\item Desta forma, se $F=G_1-G_2$, onde $G_1$ e $G_2$ são funções monótonas não-decrescentes, limitadas e contínuas à direita, então $\int \varphi(x) dF(x)$ é utilizado para denotar $$\int \varphi(x) dG_1(x)-\int \varphi(x) dG_2(x)$$.

\item Dada um intervalo qualquer $[a,b]$, define-se a integral de Lebesgue-Stieltjes de $\varphi$ em relação a $\lambda$ no intervalo $[a,b]$ como sendo $\int \varphi I_{[a,b]}d\lambda$ e denota-se por $\int_{a}^{b}\varphi d\lambda$.
\end{itemize}
\end{block}
\end{frame}
%
\begin{frame}{Propriedades da Integral de Lebesgue-Stieltjes}
%\baselineskip=13pt
\vspace{-0.2cm}
%\begin{block}{Propriedades da Integral de Lebesgue-Stieltjes}
	\begin{enumerate}
\item[P1.] Quando o integrando é contínuo, a integral de Lebesgue-Stieltjes torna-se uma integral de Riemman-Stieltjes.

\item[P2.] $\int_{a}^{b} dF=F(b)-F(a)$. Análoga ao teorema fundamental do cálculo: $\int_{a}^{b}\varphi'(x)dx=\varphi(b)-\varphi(a)$, onde $\varphi(x)$ é a derivada de $\varphi$.

\item[P3.] Linearidade no integrando e no integrador. Se $\varphi(x)=\alpha f(x)+\beta g(x)$, temos
$$\int \varphi dF=\alpha\int fdF+\beta\int gdF,$$
e para $H(x)=\alpha F(x)+\beta G(x)$, temos
$$\int \varphi dH=\alpha\int \varphi dF+\beta\int \varphi dG.$$

\item[P4.] Aditividade. Se $-\infty\leq a<b<c\leq\infty$, então
$$\int_{a}^{c} \varphi dF=\int_{a}^{b} \varphi dF+\int_{b}^{c} \varphi dF.$$

\item[P5.] Se $F$ for a função de distribuição de uma variável aleatória discreta, ou seja, se
$$F(x)=\sum_{i=1}^{\infty}p_iU(x-x_i),$$
onde $P(X=x_i)=p_i$ e $\sum_{i=1}^{\infty}p_i=1$, então
$$\int \varphi dF=\sum_{i=1}^{\infty}p_i\varphi(x_i).$$
%
\end{enumerate}
%
%\end{block}
\end{frame}
%
\begin{frame}
%\frametitle{\textbf{Propriedades da Integral de Lebesgue-Stieltjes}}
%\baselineskip=13pt
%\begin{block}{}
%
\begin{enumerate}


\item[P6.] Se $F$ for a função de distribuição de uma variável aleatória contínua, tendo densidade $f$, temos $\frac{dF(x)}{dx}=f(x)$ em quase toda parte, e consequentemente,
$$\int \varphi(x) dF(x)=\int \varphi(x)f(x)dx.$$

\item[P7.] No caso de uma distribuição geral $F$, vimos que $F$ pode ser decomposta em suas partes discreta, contínua e singular da seguinte forma $F=F_d+F_{ac}+F_s$, então por linearidade do integrador:
\begin{eqnarray}
& & \int \varphi(x) dF(x)=\int \varphi(x) dF_d(x)
\nonumber \\
& & +\int \varphi(x) dF_{ac}(x)+\int \varphi(x) dF_s(x).\nonumber
\end{eqnarray}

Se a parte singular for nula, $F_s(x)=0, \forall x$, então:
$$\int \varphi(x) dF(x)=\sum_i \varphi(x_i) p_i+\int \varphi(x)f(x)dx,$$
onde $p_i$ é o salto de $F$ em $x_i$ e $f$ é a derivada de $F$.

\end{enumerate}
%
%\end{block}
\end{frame}
%
%\begin{frame}
%\frametitle{\textbf{Propriedades da Integral de Lebesgue-Stieltjes}}
%\baselineskip=13pt
%\begin{block}{}
%
%\begin{enumerate}
%
%

%\end{enumerate}
%
%\end{block}
%\end{frame}
%
\begin{frame}
%\frametitle{\textbf}}
%\baselineskip=13pt
\begin{block}{Esperança - Caso Geral}
%
\begin{itemize}
\item Agora motivar a definição da Esperança no caso geral. 
\item Consideremos uma sequência $\{P_1,P_2,\ldots\}$ de partições de $[0,\infty)$ onde $P_n=\{y_1,y_2,\ldots,y_n\}$, $y_n\rightarrow\infty$,  $P_{i+i}$ é um refinamento de $P_i$, e a norma de $P_n$ tende a zero quando $n\rightarrow\infty$.

\item Dada uma variável aleatória não-negativa qualquer $X$ e uma partição $P_n$ desta sequência, definamos uma outra variável aleatória $Y$ discreta que aproxima $X$ assumindo o valor $y_{i}$ quando $y_{i}\leq X<y_{i+1}$ e $Y=y_n$ se $X\geq y_n$, ou seja, $Y=\sum_{i=1}^{n-1}y_{i} I_{[y_{i}\leq X<y_{i+1}]}+y_nI_{[X\geq y_n]}$. 

\item Como $Y$ é discreta temos que sua esperança é dada por
\begin{eqnarray}
& & EY=\sum_{i=1}^{n} y_{i}P(Y=y_{i}) \nonumber\\
& & =\sum_{i=1}^{n-1} y_{i}P(y_{i}\leq X<y_{i+1})+y_nP(X\geq y_n).\nonumber
\end{eqnarray}
\end{itemize}
\end{block}
\end{frame}
%
\begin{frame}
%\frametitle{\textbf{Esperança - Caso Geral}}
%\baselineskip=13pt
\begin{block}{}
\begin{itemize}
\item Note que esta esperança é uma soma de Lebesgue em relação a partição $P_n$ com integrando X e função integradora dada pela medida de probabilidade $P$. 

\item Note que a medida que pegamos partições mais refinadas na sequência, $Y$ se torna cada vez uma melhor aproximação para $X$. 

\item Uma vez que os valores de $X$ e $Y$ ficam cada vez mais próximos  é intuitivo requerer que nossa definição de esperança (média) $EX$ seja igual ao limite de $EY$ quando $n\rightarrow\infty$, ou seja
\begin{eqnarray}
& & EX=\lim_{n\rightarrow \infty}\sum_{i=1}^{n} y_{i}P(Y=y_{i}) \nonumber\\
& & =\lim_{n\rightarrow \infty} \sum_{i=1}^{n-1} y_{i}P(y_{i}\leq X<y_{i+1})+y_nP(X\geq y_n)
\nonumber\\
& & =\int XdP.\nonumber
\end{eqnarray}
\end{itemize}
\end{block}
\end{frame}
%
\begin{frame}
%\frametitle{\textbf{Esperança - Caso Geral}}
%\baselineskip=13pt
%\begin{block}{}
%
Logo, $EX$ é definida como sendo a integral de Lebesgue-Stieltjes de $X$ em relação a medida de probabilidade $P$, ou similarmente, $EX=\int XdF$, onde $F$ é a função de distribuição acumulada de $X$. No caso geral, temos a seguinte definição
\begin{defi}
Se $X$ é uma variável aleatória com função de distribuição $F$, então sua esperança é dada pela fórmula
$$EX=\int XdF=\int_{-\infty}^0 X dF + \int_0^{\infty}XdF,$$
desde que pelo menos uma das integrais seja finita. Em caso as duas
integrais não sejam finitas, a esperança não existe. Caso $EX$ seja finita, diz-se que $X$ é integrável.
\end{defi}
%
%\end{block}
%\end{frame}
%
%\begin{frame}
%\frametitle{\textbf{Esperança - Caso Geral}}
%\baselineskip=13pt
%\begin{block}{}
%
%
Pela Propriedade P7 da integral de Lebesgue-Stieltjes, temos que se $F=F_d+F_{ac}+F_s$, então
$$EX=\int XdF=\sum_ix_ip_i + \int xf(x)dx + \int xdF_s(x),$$
onde $p_i$ é o salto de $F$ em $x_i$ e $f$ é a derivada de $F$. Como a parte singular costuma ser nula, na prática a esperança reduz-se a uma série e/ou uma integral imprópria, usualmente de Riemann se $f$ for integrável no sentido de Riemann.

%\end{block}
\end{frame}
%
\begin{frame}
%\frametitle{\textbf{Exemplo}}
%\baselineskip=13pt
\begin{exem}[Variável Mista]
%
%
%%\begin{example}
Considere uma variável aleatória $Y$ com função de distribuição $F$, tal que
$$F(x)=
\left\{
\begin{array}{ll}
0 & \mbox{, se $x<0$} \\
x & \mbox{, se $0\leq x<1/2$} \\
1 & \mbox{, se $x\geq 1/2$.}
\end{array}
\right.$$
%
%\end{exem}
%\end{frame}
%
%\begin{frame}
%\frametitle{\textbf{Exemplo}}
%\baselineskip=13pt
%\begin{block}{Variável Mista}
%
%
Decompondo em parte discreta e contínua tem-se
$$F_d(x)=
\left\{
\begin{array}{ll}
0 & \mbox{, se $x<1/2$} \\
1/2 & \mbox{, se $x\geq 1/2$,}
\end{array}
\right.$$
e
$$F_{ac}(x)=
\left\{
\begin{array}{ll}
0 & \mbox{, se $x<0$} \\
x & \mbox{, se $0\leq x<1/2$} \\
1/2 & \mbox{, se $x\geq 1/2$.}
\end{array}
\right.$$
Portanto,
$$EY=\frac{1}{2}P(Y=\frac{1}{2})+\int_{0}^{1/2}ydy=\frac{1}{4}+\frac{1}{8}=\frac{3}{8}.$$
\end{exem}
%
%\end{block}
\end{frame}
%
\begin{frame}
%\frametitle{\textbf{Exemplo}}
%\baselineskip=13pt
\begin{exem}[Uniforme]
%
Se $X\sim U(a,b)$, então $X$ possui densidade igual a $f(x)=\frac{1}{b-a}$ se $x\in(a,b)$, e $f(x)=0$, caso contrário. Logo, temos
que sua esperança é dada por:
$$EX=\int_{a}^{b}\frac{x}{b-a}dx=\frac{a+b}{2}.$$
%
\end{exem}
%
\begin{exem}[Exponencial]
%
Se $X\sim Exp(\lambda)$, então $X$ possui densidade igual a $f_X(x)=\lambda e^{-\lambda x}U(x)$. Logo, temos
que sua esperança é dada por:
\begin{eqnarray}
& & EX=\int_{0}^{\infty}x\lambda e^{-\lambda x}dx=-xe^{-\lambda x}|_{0}^{\infty}+\int_{0}^{\infty}e^{-\lambda x}dx \nonumber \\
& &  =\frac{-e^{-\lambda x}}{\lambda}|_{0}^{\infty}=\frac{1}{\lambda}.\nonumber
\end{eqnarray}

\end{exem}
\end{frame}
%
\begin{frame}
%\frametitle{\textbf{Exemplo}}
%\baselineskip=13pt
\begin{exem}[Normal]
%
Se $X\sim {\cal N}(\mu,\sigma)$, então $X$ possui densidade igual a $f_X(x)=\frac{1}{\sigma\sqrt{2\pi}}e^{\frac{-(x-\mu)^2}{2\sigma^2}}$. Logo, temos
que sua esperança é dada por:
$$EX=\int_{-\infty}^{\infty}x\frac{1}{\sigma\sqrt{2\pi}}e^{\frac{-(x-\mu)^2}{2\sigma^2}}dx.$$
Fazendo a mudança de variável $y=\frac{x-\mu}{\sigma}$, temos

\begin{eqnarray}
\hspace{-0.7cm} & & EX=\int_{-\infty}^{\infty}\frac{\sigma y +\mu}{\sqrt{2\pi}}e^{\frac{-y^2}{2}}dy
\nonumber  =\int_{-\infty}^{\infty}\frac{\sigma y }{\sqrt{2\pi}}e^{\frac{-y^2}{2}}dy+\int_{-\infty}^{\infty}\frac{\mu}{\sqrt{2\pi}}e^{\frac{-y^2}{2}}dy=0+\mu=\mu.\nonumber
\end{eqnarray}
%
\end{exem}
%\end{frame}
%
%\begin{frame}
%\frametitle{\textbf{Exemplo}}
%\baselineskip=13pt
\begin{exem}[Cauchy]
%
Se $X\sim Cauchy(a)$, então $X$ possui densidade igual a $f_X(x)=\frac{1}{\pi}\cdot\frac{a}{a^2+x^2}$. Neste caso $X$ não é integrável, ou seja $EX$ não está definida, pois:
$$\int_{-\infty}^{0}\frac{x}{\pi}\cdot\frac{a}{a^2+x^2}dx=\frac{a}{2\pi}\log(a^2+x^2)|_{-\infty}^{0}=-\infty, \mbox{ e}$$
$$\int_{0}^{\infty}\frac{x}{\pi}\cdot\frac{a}{a^2+x^2}dx=\frac{a}{2\pi}\log(a^2+x^2)|_{0}^{\infty}=\infty.$$

\end{exem}
\end{frame}
%
\begin{frame}
%\frametitle{\textbf{}}
%\baselineskip=13pt
\begin{block}{Interpretação Geométrica da Esperança}
%
%
Por definição, $EX=\int xdF(x)$, ou seja, $EX$ é a integral da diferencial $xdF$. Mas $xdF$ é uma diferencial de área. \begin{itemize}
\item Para $x>0$, $xdF$ é uma diferencial da área da região compreendida entre as curvas $x=0$, $y=1$, e $y=F(x)$ no plano Euclideano, cuja área total é dada por $\int_{0}^{\infty}(1-F(x))dx$.
\item Para $x<0$, $-xdF$ é uma diferencial da área da região compreendida entre as curvas $x=0$, $y=0$, e $y=F(x)$ no plano Euclideano, cuja área total é dada por $\int_{-\infty}^{0}F(x)dx$.
\item Logo, $EX=\int_{0}^{\infty}(1-F(x))dx-\int_{-\infty}^{0}F(x)dx$.
\end{itemize}
%
\end{block}
\end{frame}
%
\begin{frame}
%\frametitle{\textbf{Interpretação Geométrica da Esperança}}
%\baselineskip=13pt
%\begin{block}{}
%
%
Formalmente, podemos provar isso da seguinte maneira. A prova será dividida em duas etapas:
\begin{enumerate}
\item[(a)] $\int_{0}^{\infty}xdF(x)=\int_{0}^{\infty}(1-F(x))dx$ e
\item[(b)] $\int_{-\infty}^{0}xdF(x)=-\int_{-\infty}^{0}F(x)dx$.
\end{enumerate}
Comecemos provando (b). Utilizando integração por partes, temos que $\forall a<0$,
$$ \int_{a}^{0}xdF(x)=-aF(a)-\int_{a}^{0}F(x)dx  =\int_{a}^{0}[F(a)-F(x)]dx.
$$
%
%\end{block}
%\end{frame}
%
%\begin{frame}
%\frametitle{\textbf{Interpretação Geométrica da Esperança}}
%\baselineskip=13pt
%\begin{block}{}
%
Como $F(a)\geq 0$ e $a<0$, temos
$$\int_{a}^{0}xdF(x)\geq -\int_{a}^{0}F(x)dx.$$
Como a desigualdade é válida para todo $a<0$, temos que tomando o limite quando $a\rightarrow -\infty$
$$\int_{-\infty}^{0}xdF(x)\geq -\int_{-\infty}^{0}F(x)dx.$$
Por outro lado, seja $\lambda<0$. Se $a<\lambda$, então
\begin{eqnarray}
& & \int_{a}^{0}[F(a)-F(x)]dx\leq \int_{\lambda}^{0}[F(a)-F(x)]dx\nonumber \\
& & =F(a)(-\lambda)-\int_{\lambda}^{0}F(x)dx. \nonumber
\end{eqnarray}

%\end{block}
\end{frame}
%
\begin{frame}
%\frametitle{\textbf{Interpretação Geométrica da Esperança}}
%\baselineskip=13pt
%\begin{block}{}
%
Portanto, tomando o limite quando $a\rightarrow -\infty$, temos que
$$\int_{-\infty}^{0}xdF(x)\leq -\int_{\lambda}^{0}F(x)dx.$$
Como isto é válido para todo $\lambda<0$, tomando o limite quando $\lambda\rightarrow -\infty$, temos
$$\int_{-\infty}^{0}xdF(x)\leq -\int_{-\infty}^{0}F(x)dx,$$
como queríamos demonstrar.
%
%\end{block}
%\end{frame}
%
%\begin{frame}
%\frametitle{\textbf{Interpretação Geométrica da Esperança}}
%\baselineskip=13pt
%\begin{block}{}
%
%
Para parte (a), utilizando integração por partes, temos que $\forall b>0$,
$$\int_{0}^{b}xdF(x)=bF(b)-\int_{0}^{b}F(x)dx=\int_{0}^{b}[F(b)-F(x)]dx.$$
Como $F(b)\leq 1$ e $1-F(x)\geq 0$, temos
$$\int_{0}^{b}xdF(x)=\int_{0}^{b}[F(b)-F(x)]dx\leq \int_{0}^{\infty}[1-F(x)]dx.$$
Como a desigualdade é válida para todo $b>0$, temos que tomando o limite quando $b\rightarrow\infty$
$$\int_{0}^{\infty}xdF(x)\leq\int_{0}^{\infty}[1-F(x)]dx.$$

%\end{block}
\end{frame}

\begin{frame}
%\frametitle{\textbf{Interpretação Geométrica da Esperança}}
%\baselineskip=13pt
%\begin{block}{}

Por outro lado, seja $\lambda>0$. Se $b>\lambda$, então
\begin{eqnarray}
& & \int_{0}^{b}[F(b)-F(x)]dx\geq \int_{0}^{\lambda}[F(b)-F(x)]dx\nonumber \\
& &  =\int_{0}^{\lambda}[F(b)-1]dx+\int_{0}^{\lambda}[1-F(x)]dx \nonumber \\ & & =\lambda[F(b)-1]+\int_{0}^{\lambda}[1-F(x)]dx, \nonumber
\end{eqnarray}
e portanto, tomando o limite quando $b\rightarrow\infty$, temos que
$$\int_{0}^{\infty}xdF(x)\geq\int_{0}^{\lambda}[1-F(x)]dx.$$
Como isto é válido para todo $\lambda>0$, tomando o limite quando $\lambda\rightarrow\infty$, temos
$$\int_{0}^{\infty}xdF(x)\geq\int_{0}^{\infty}[1-F(x)]dx,$$
como queríamos demonstrar.

%\end{block}
\end{frame}
%
\begin{frame}
\frametitle{Esperança de Funções de Variáveis Aleatórias}
%\baselineskip=13pt
\begin{block}{Caso Discreto}
%
Como vimos anteriormente, se $X$ for uma variável aleatória discreta
e se $Y=H(X)$, então $Y$ também será uma variável aleatória
discreta. Consequentemente, pode-se calcular $EY$. Existem duas
maneiras de calcular $EY$ que são equivalentes.
\end{block}
\begin{defi}
Seja $X$ uma variável aleatória discreta e seja $Y=H(X)$. Se $Y$
assumir os seguintes valores $y_1,y_2,\ldots$  e se
$p(y_i)=P(Y=y_i)$, definimos: $$EY=\sum_{i=1}^{\infty}y_ip(y_i).$$
\end{defi}
%\end{frame}
%
%\begin{frame}
%\frametitle{\textbf{Esperança de Funções de Variáveis Aleatórias}}
%\baselineskip=13pt
%\begin{block}{Caso Discreto}
%
%
Conforme vimos anteriormente podemos determinar as
probabilidades $p(y_i)$ dado que sabemos a distribuição de $X$. No
entanto, podemos encontrar $EY$ sem preliminarmente encontrarmos a
distribuição de probabilidade de $Y$, partindo-se apenas do
conhecimento da distribuição de probabilidade de $X$, conforme
mostra o seguinte teorema.
%
%
%\end{block}
\end{frame}
%
\begin{frame}
%\frametitle{\textbf{Esperança de Funções de Variáveis Aleatórias}}
%\baselineskip=13pt
%\begin{block}{Caso Discreto}
%
%
\begin{teo}[Caso Discreto] \label{thm:mom_func} Seja $X$ uma variável aleatória discreta assumindo os valores $x_1,x_2,\ldots$ e
seja $Y=H(X)$. Se $p(x_i)=P(X=x_i)$, temos $$EY=E(H(X))=\sum_{i=1}^{\infty}H(x_i)p(x_i).$$ \end{teo}
%\end{block}
%
%
%\end{frame}
%
%\begin{frame}
%\frametitle{\textbf{Esperança de Funções de Variáveis Aleatórias}}
%\baselineskip=13pt
%\begin{block}{Prova do Teorema}
%
\begin{proof} 
Vamos re-ordenar o somatório $\sum_{i=1}^{\infty}H(x_i)p(x_i)$,
agrupando os termos onde $x_i$ tem a mesma imagem de acordo com a
função $H$, ou seja, sejam $x_{i1},x_{i2},\ldots$, todos os valores
$x_i$ tal que $H(x_{ij})=y_i$ para $j\geq 1$, onde $y_1,y_2,\ldots$
são os possíveis valores de $Y$. Desse modo podemos reescrever
\begin{eqnarray}
& & \sum_{i=1}^{\infty}H(x_i)p(x_i)=\sum_{i=1}^{\infty}\sum_{j=1}^{\infty}H(x_{ij})p(x_{ij})
\nonumber\\
& & =\sum_{i=1}^{\infty}y_i\sum_{j=1}^{\infty}p(x_{ij})=\sum_{i=1}^{\infty}y_ip(y_i)=EY.\nonumber
\end{eqnarray}
\end{proof}
%
%
%\end{block}
\end{frame}
%
\begin{frame}
%\frametitle{\textbf{Exemplo}}
%\baselineskip=13pt
%\begin{block}{}
%
%
\begin{exem}
Suponha que $X$ é uma variável aleatória Poisson com parâmetro
$\lambda$. Seja $Y=X^2$, vamos calcular $EY$. Utilizando o
Teorema~\ref{thm:mom_func}, temos

\begin{eqnarray}
& &
EY=\sum_{k=0}^{\infty}k^2e^{-\lambda}\frac{\lambda^k}{k!}\nonumber\\
& & =\sum_{k=1}^{\infty}k^2e^{-\lambda}\frac{\lambda^k}{k!}=\sum_{k=1}^{\infty}k(k-1)e^{-\lambda}\frac{\lambda^k}{k!}+\sum_{k=1}^{\infty}ke^{-\lambda}\frac{\lambda^k}{k!}
\nonumber \\
&
&=\lambda^2\sum_{k=2}^{\infty}e^{-\lambda}\frac{\lambda^{k-2}}{(k-2)!}+\lambda=\lambda^2+\lambda.\nonumber
\end{eqnarray}
\end{exem}

%\end{block}
\end{frame}
%
\begin{frame}
%\frametitle{\textbf{Esperança de Funções de Variáveis Aleatórias}}
%\baselineskip=13pt
\begin{block}{Caso Discreto Vetorial}
%
%
Também podemos estender este resultado para o caso de uma função
real de um vetor aleatório. Neste caso, se $Y=H(\vec{X})$, temos que
$EY=\sum_i H(\vec{x_i})p_{\vec{X}}(\vec{x_i})$, onde os $\vec{x_i}$ são os
valores assumidos pelo vetor aleatório $\vec{X}$.

\end{block}
%\end{frame}
%
%\begin{frame}
%\frametitle{\textbf{Esperança de Funções de Variáveis Aleatórias}}
%\baselineskip=13pt
\begin{block}{Caso Geral}
%
No caso de uma variável aleatória qualquer $X$ também podemos calcular a esperança de uma função $Y=\varphi(X)$ de forma similar.
\end{block}
\begin{teo}
Seja $X$ uma variável aleatória qualquer, $Y=\varphi(X)$ uma outra variável aleatória, então
$$EY=\int ydF_Y(y)=\int \varphi(x)dF_X(x),$$
desde que estas integrais existam.
\end{teo}

\begin{proof}
A prova no caso geral envolve Teoria da Medida e será omitida.
\end{proof}

\end{frame}
%
\begin{frame}
%\frametitle{\textbf{Esperança de Funções de Variáveis Aleatórias}}
%\baselineskip=13pt
\begin{block}{Caso Vetorial}
%
Uma fórmula análoga também é válida quando consideramos funções de vetores aleatórios.
\end{block}

\begin{teo}Seja $\vec{X}$ um vetor aleatório e $Y=\varphi(\vec{X})$ uma variável aleatória. Então,
$$EY=\int ydF_Y(y)=\int \varphi dF_{\vec{X}}.$$
\end{teo}
%
%
%\end{frame}
%
%\begin{frame}
%\frametitle{\textbf{}}
%\baselineskip=13pt
\begin{block}{Propriedades da Esperança}
%
As seguintes propriedades são aplicações imediatas da definição de
esperança:
\begin{itemize}
\item[1.] $P(X=c)=1\Rightarrow EX=c$.
\item[2.] $P(X\geq 0)=1\Rightarrow EX\geq 0$.
\item[3.] $E(aX)=aEX$, onde $a \in \mathbb{R}$ (segue da esperança de uma função
de v. a.)
\end{itemize}
%
%\end{block}
%\end{frame}
%%
%%\begin{frame}
%%\frametitle{\textbf{Propriedades da Esperança}}
%%\baselineskip=13pt
%%\begin{block}{}
%%
%%\begin{itemize}


\end{block}
\end{frame}
%
\begin{frame}
%\frametitle{\textbf{Propriedades da Esperança}}
%\baselineskip=13pt
\begin{itemize}
\item[4.] $E(X+Y)=EX+EY$.
No caso discreto, note que
$$ E(X+Y)=\sum_i\sum_j(x_i+y_j)p(x_i,y_j) =\sum_i
x_i\sum_jp(x_i,y_j)+\sum_i\sum_jy_jp(x_i,y_j)$$ $$=\sum_i x_ip(x_i)+\sum_j y_j\sum_ip(x_i,y_j)=EX+\sum_j
y_jp(y_j)=EX+EY.
$$


\end{itemize}
No caso geral, temos que
$$E(X+Y)=E(\varphi(X,Y))=\int\int(x+y)dF_{X,Y}(x,y),$$
e pela linearidade da integral obtemos
\begin{eqnarray}
& & E(X+Y)=\int\int xdF_{X,Y}(x,y)+\int\int ydF_{X,Y}(x,y)=EX+EY.\nonumber
\end{eqnarray}
%
%
\begin{corol}
$E(\sum_i^n a_iX_i)=\sum_i^n a_iEX_i.$
\end{corol}
%
\begin{proof}
Aplicação das duas propriedades anteriores e indução matemática.
\end{proof}
%
%\end{block}
\end{frame}
%
\begin{frame}
%\frametitle{\textbf{Propriedades da Esperança}}
%\baselineskip=13pt
%\begin{block}{}
%
%
\begin{itemize}
\item[5.] $P(X\geq Y)=1\Rightarrow EX\geq EY$.
Propriedade 5 segue da propriedades 2 e do corolário anterior, pois
$$P(X\geq Y)=P(X-Y\geq 0),$$
o que, pela propriedade 2, implica que $E(X-Y)\geq 0$. Pelo corolário, temos que $E(X-Y)=EX-EY$, ou seja podemos concluir
que $EX-EY\geq 0$.
%
\item[6.] Se $\{X_1,\ldots,X_n\}$ são variáveis aleatórias mutuamente
independentes, então $$E(\prod_{i=1}^{n}X_i)=\prod_{i=1}^{n}EX_i.$$
Provaremos esta propriedade nos casos discreto e contínuo. No caso discreto, note que
\begin{eqnarray}
& & E(\prod_{i=1}^{n}X_i)=\sum_{i_1}\cdots\sum_{i_n}x_{i_1}\cdots
x_{i_n}p(x_{i_1},\ldots,x_{i_n}) \nonumber \\
& & =\sum_{i_1}\cdots\sum_{i_n}x_{i_1}\cdots
x_{i_n}\prod_{j=1}^{n}p(x_{i_j})\nonumber\\
& & =\sum_{i_1}x_{i_1}p(X_{i_1})\cdots\sum_{i_n}
x_{i_n}p(x_{i_n})=\prod_{i=1}^n EX_i. \nonumber
\end{eqnarray}

\end{itemize}
%
%\end{block}
\end{frame}
%
\begin{frame}
%\frametitle{\textbf{Propriedades da Esperança}}
%\baselineskip=13pt
%\begin{block}{}
%
No caso contínuo, temos que $f_{\vec{X}}(\vec{x})=\prod_{i=1}^{n}f_{X_i}(x_i)$, logo
\begin{eqnarray}
& & E(\prod_{i=1}^{n}X_i)=\int\cdots\int x_1\cdots x_n f_{\vec{X}}(\vec{x})dx_1\cdots dx_n \nonumber \\
& &=\int\cdots\int \prod_{i=1}^{n}x_if_{X_i}(x_i)dx_1\cdots dx_n \nonumber \\
& & =\prod_{i=1}^{n}\int x_if_{X_i}(x_i)dx_i=\prod_{i=1}^n EX_i. \nonumber
\end{eqnarray}

De maneira análoga, pode-se provar a seguinte generalização deste resultado:
se $\{X_1,\ldots,X_n\}$ são variáveis aleatórias mutuamente
independentes, então $$E(\prod_{i=1}^{n}G(X_i))=\prod_{i=1}^{n}EG(X_i)$$.


%\end{block}
\end{frame}
%
\begin{frame}
%\frametitle{\textbf{Propriedades da Esperança}}
%\baselineskip=13pt
%\begin{block}{}
%
\begin{itemize}
%
%
\item[7.] Se $Y$ for uma variável aleatória que assume
valores inteiros não-negativos, temos que
$$EY=\sum_{k=1}^{\infty}kP(Y=k)=\sum_{k=1}^{\infty}\sum_{j=1}^{k}P(Y=k),$$
trocando a ordem dos somatórios:
\begin{equation}
EY=\sum_{j=1}^{\infty}\sum_{k=j}^{\infty}P(Y=k)=\sum_{j=1}^{\infty}P(Y\geq
j).\nonumber\end{equation}
%
%\end{description}
%
%\end{block}
%\end{frame}
%
%\begin{frame}
%\frametitle{\textbf{Propriedades da Esperança}}
%\baselineskip=13pt
%\begin{block}{}
%
%\begin{itemize}
%
%
\item[8.] (Desigualdade de Jensen) Seja $\varphi$ uma função mensurável e convexa definida na reta. Se $X$ é integrável, então $E\varphi(X)\geq \varphi(EX)$.

\begin{proof}
Pela convexidade de $\varphi$, dado algum ponto $(x_0,\varphi(x_0)$ do gráfico de $\varphi$, existe uma reta que passa por esse ponto e fica sempre abaixo do gráfico de $\varphi$, ou seja, existe algum $\lambda$ tal que
$$\varphi(x)\geq \varphi(x_0)+\lambda(x-x_0),\forall x.$$
Logo, pela monotonicidade e linearidade da esperança, temos
$$E\varphi(X)\geq \varphi(x_0)+\lambda(EX-x_0).$$
Em particular, para $x_0=EX$, temos $E\varphi(X)\geq \varphi(EX)$.
\end{proof}

\end{itemize}

\end{frame}
%
\begin{frame}{Integrabilidade}
%\frametitle{\textbf}
%\baselineskip=13pt
%\begin{block}{}
%
%
%
O próximo Lema estabelece um critério para integrabilidade de variáveis aleatórias.
\begin{lema} \label{lem:integ} Seja $X$ uma variável aleatória qualquer.
Então,
$\sum_{n=1}^{\infty}P(|X|\geq n)\leq E|X|\leq 1+\sum_{n=1}^{\infty}P(|X|\geq n),$
e, portanto, $X$ é integrável se, e somente se,
$$\sum_{n=1}^{\infty}P(|X|\geq n)<\infty.$$ 
\end{lema}
%
%\end{block}
%\end{frame}
%
%\begin{frame}
%\frametitle{\textbf{Prova do Lema}}
%\baselineskip=13pt
%\begin{block}{}
%
%
\begin{proof} Se $x\geq 0$, seja $\lfloor x\rfloor$ a parte inteira de $x$. Então,
a variável aleatória $\lfloor |X|\rfloor$ assume o valor $k$ quando
$k\leq |X|<k+1$ e $0\leq \lfloor |X|\rfloor\leq |X|\leq \lfloor
|X|\rfloor +1$, então pela monotonicidade e linearidade da esperança
temos:
$0\leq E\lfloor |X|\rfloor\leq E|X|\leq 1+E\lfloor
|X|\rfloor.$

Como $\lfloor |X|\rfloor$ é uma variável aleatória que só assume
valores inteiros não-negativos, temos
$$E\lfloor |X|\rfloor=\sum_{n=1}^{\infty}P(\lfloor |X|\rfloor\geq n)=\sum_{n=1}^{\infty}P(|X|\geq n),$$
logo
$$\sum_{n=1}^{\infty}P(|X|\geq n)\leq E|X|\leq 1+\sum_{n=1}^{\infty}P(|X|\geq n).$$
\end{proof}
\end{frame}
%
\begin{frame}
%\frametitle{\textbf{Integrabilidade}}
%\baselineskip=13pt
%\begin{block}{}
%
%
Se $X^+=\max(X,0)$ e $X^-=-\min(X,0)$, temos que $X=X^+-X^-$ e $|X|=X^++X^-$. Por definição, temos que $EX<\infty$ se, e somente  se, $EX^+<\infty$ e $EX^-<\infty$. Portanto, vemos que $EX<\infty$ se, e somente se, $E|X|<\infty$. De forma análoga, pode-se concluir que $E\varphi(X)<\infty$ se, e somente se, $E|\varphi(X)|<\infty$ para qualquer função mensurável $\varphi$. O próximo teorema nos dá um outro critério para integrabilidade de uma variável aleatória.
%
\begin{teo}
Sejam $X$ e $Y$ variáveis aleatórias tais que $Y\geq 0$, $Y$ é integrável, e $|X|<Y$. Então, $X$ é integrável.
\end{teo}
%
\begin{proof}
Note que $0\leq |X|\leq Y$ implica que $0\leq E|X|\leq EY$. Portanto, se $EY<\infty$, temos que $E|X|<\infty$, o que por sua vez implica que $EX<\infty$.
\end{proof}
%
%\end{block}
\end{frame}
%
\section{Momentos}
\begin{frame}
\frametitle{Momentos}
%\baselineskip=13pt
%\begin{block}{}
%
Momentos dão informações parciais sobre a medida de probabilidade
$P$, a função de distribuição acumulada, ou a função probabilidade
de massa de uma variável aleatória $X$. Momentos de $X$ são
esperanças de potências de $X$.
%
\begin{defi}
Para qualquer inteiro não-negativo $n$, o $n$-ésimo momento da
variável aleatória $X$ é $EX^n$, se esta esperança existe.
\end{defi}
%
Vimos anteriormente que o segundo momento de uma variávelaleatória Poisson com parâmetro $\lambda$ é dado por: $\lambda^2+\lambda$. Vamos agora calcular o segundo momento de uma variável aleatória $X$ Binomial com parâmetros $n$ e $p$:

\begin{eqnarray}
& &
EX^2=\sum_{k=0}^{n}k^2\binom{n}{k}p^k(1-p)^{n-k}=\sum_{k=1}^{n}k^2\frac{n!}{k!(n-k)!}p^k(1-p)^{n-k}\nonumber\\
& & =\sum_{k=1}^{n}k(k-1)\frac{n!}{k!(n-k)!}p^k(1-p)^{n-k}+\sum_{k=1}^{n}k\frac{n!}{k!(n-k)!}p^k(1-p)^{n-k}\nonumber\\
& & =n(n-1)p^2\sum_{k=2}^{n}\frac{(n-2)!}{(k-2)!(n-k)!}p^{k-2}(1-p)^{n-k}+np\nonumber\\
& & =n(n-1)p^2\sum_{j=0}^{m}\frac{(m)!}{(j)!(m-j)!}p^{j}(1-p)^{m-j}+np=n(n-1)p^2+np.\nonumber
\end{eqnarray}

%\end{block}
\end{frame}
%
\begin{frame}{Existência de momentos}
%\frametitle{\textbf{Momentos}}
%\baselineskip=13pt
%\begin{block}{Existência}
%
%
\begin{teo}
Se o $k$-ésimo momento de uma variável aleatória for finito, então todos os momentos de ordem menores do que $k$ também são finitos.
\end{teo}
%
\begin{proof}
Por hipótese, temos que $E|X^k|<\infty$, logo $E(1+|X^{k}|)<\infty$. Como para qualquer $j$ tal que $0<j<k$, $|X^{j}|\leq 1+|X^{k}|$, e $1+|X^{k}|$ é integrável, temos que $|X^{j}|$ também é integrável.
\end{proof}
%
%\commentout{
%\thm 
\begin{teo}Se $j<k$, então $$|EX^k|<\infty\Rightarrow |EX^j|<\infty$$
\end{teo}

\begin{block}{Demonstração}Vamos provar apenas para o caso em que $X$ é uma variável
aleatória discreta. Por hipótese, temos que existem constantes
finitas $c_1$ e $c_2$ tais que:
$$c_1>\sum_{i:x_i\geq 0}x_i^kp(x_i)\mbox{, e }c_2>|\sum_{i:x_i<0}x_i^kp(x_i)|.$$

\end{block}
\end{frame}

\begin{frame}
\begin{block}{}
	Então, para $j<k$
\begin{eqnarray}
& & c_1>\sum_{i:x_i\geq 0}x_i^kp(x_i)\geq \sum_{i:x_i\geq
1}x_i^kp(x_i)\geq \sum_{i:x_i\geq 1}x_i^jp(x_i)\nonumber\\
& & \geq \sum_{i:x_i\geq 0}x_i^jp(x_i)-\sum_{i:0\leq x_i\leq
1}x_i^jp(x_i)\geq \sum_{i:x_i\geq 0}x_i^jp(x_i)-\sum_{i:0\leq
x_i\leq 1}p(x_i)=\sum_{i:x_i\geq 0}x_i^jp(x_i)-1\nonumber
\end{eqnarray}
Portanto,
$$0\leq \sum_{i:x_i\geq 0}x_i^jp(x_i)<c_1+1<\infty.$$
Um cálculo similar prova que $\sum_{i:x_i< 0}x_i^jp(x_i)$ é finito.
\end{block}
%
Portanto, temos que {\em momentos de ordem superiores finitos
implicam momentos de ordem inferiores finitos.}
%}
%
%\end{block}

Vamos agora enunciar dois teoremas importantes que tratam
da convergência de esperanças de variáveis aleatórias. Neste caso, estaremos tratando de convergência pontual de variáveis aleatórias, ou seja, $X_n\rightarrow X$ se, e somente se, $X_n(\omega)\rightarrow X(\omega)$ para todo $\omega\in \Omega$. Mais adiante, veremos outras noções de convergência de variáveis aleatórias.
\end{frame}
%
\begin{frame}{Convergência Monótona e Convergência Dominada}
%\frametitle{\textbf}
%\baselineskip=13pt
%\begin{block}{}
%
%
%Vamos agora enunciar dois teoremas importantes que tratam
%da convergência de esperanças de variáveis aleatórias. Neste caso, estaremos tratando de convergência pontual de variáveis aleatórias, ou seja, $X_n\rightarrow X$ se, e somente se, $X_n(w)\rightarrow X(w)$ para todo $w\in \Omega$. Mais adiante, veremos outras noções de convergência de variáveis aleatórias.
%
\begin{teo}{Teorema da Convergência Monótona.} Sejam
$X,X_1,X_2,\ldots$ variáveis aleatórias. Se $0\leq X_n\uparrow X$,
então, $EX_n\uparrow EX$. \end{teo}

\begin{teo}{Teorema da Convergência Dominada.} Sejam
$Y,X,X_1,X_2,\ldots$ variáveis aleatórias. Considere que $Y$ seja
integrável, $|X_n|\leq Y$ e $X_n\rightarrow X$. Assim $X$ e $X_n$
são integráveis e $EX_n\rightarrow EX$. \end{teo}
%
%\end{block}
%\end{frame}
%
%\begin{frame}
%\frametitle{\textbf{Exemplo}}
%\baselineskip=13pt
%O próximo exemplo mostra que 
\begin{nota} Nem sempre $X_n\rightarrow X\Rightarrow
EX_n\rightarrow EX$.
\end{nota}
%\begin{exem}
%
%
%\begin{example}
%Seja $Y\thicksim U(0,1)$. Considere a seguinte sequência
%$\{X_1,X_2,\ldots\}$ de variáveis aleatórias: $X_n(\omega)=n$ se
%$Y(\omega)\in (0,1/n)$ e $X_n(\omega)=0$ em caso contrário. Então,
%temos que $X_n(\omega)\rightarrow 0,\forall\omega$. Mas, $EX_n=1\ne
%0=E0$, ou seja, $EX_n\nrightarrow 0$.
%\end{exem}
%
%
%\end{block}
%\end{frame}
%
%\begin{frame}
%\frametitle{\textbf}
%\baselineskip=13pt
%\begin{block}{}
%
\begin{defi}[Momentos Centrais]
Se $X$ é uma variável aleatória seu $n$-ésimo momento central é:
$E(X-EX)^n$, se esta esperança existir.
\end{defi}
\end{frame}
%
\begin{frame}
Note que o primeiro momento central é zero, pois
$E(X-EX)=EX-EEX=EX-EX=0$. O segundo momento central é conhecido como
{\em variância} e denota-se por $Var X$. A variância pode ser também
calculada por:
\begin{eqnarray}
& & Var X=E(X-EX)^2=E(X^2-2XEX+(EX)^2)\nonumber\\
& & =EX^2-2E(XEX)+E((EX)^2)
\nonumber\\
& & =EX^2-2(EX)^2+(EX)^2=EX^2-(EX)^2.\nonumber
\end{eqnarray}

\begin{block}{Existência}
Do Teorema
Binomial e da linearidade da esperança, temos
$$E(X-EX)^n=\sum_{k=0}^{n}\binom{n}{k}(-EX)^{n-k}EX^k$$
e
\begin{eqnarray}
& & EX^n=E(X-EX+EX)^n \nonumber\\
& & =\sum_{k=0}^{n}\binom{n}{k}(EX)^{n-k}E(X-EX)^k.\nonumber
\end{eqnarray}

Como um corolário, temos que o $n$-ésimo momento central é finito se,
e somente se, o $n$-ésimo momento é finito.

\end{block}
\end{frame}
%
\begin{frame}
%\frametitle{\textbf{Exemplo}}
%\baselineskip=13pt
%\begin{block}{}
%
%
\begin{exem}
Considere uma variável aleatória $X$ tal que
\begin{eqnarray}
& & P(X=m-a)=P(X=m+a)=\frac{1}{2}
\nonumber\\
& & \Rightarrow EX^k=\frac{1}{2}[(m-a)^k+(m+a)^k].\nonumber
\end{eqnarray}
$$EX=m,EX^2=\frac{1}{2}[2m^2+2a^2]=m^2+a^2,Var X=a^2.$$
Este exemplo, mostra que podemos encontrar uma variável aleatória
bem simples possuindo qualquer esperança e variância
predeterminadas.
\end{exem}
%
%\end{block}
%\end{frame}
%
%\begin{frame}
%\frametitle{\textbf{}}
%\baselineskip=13pt
\begin{block}{Desvio-Padrão}
%
%
O {\em desvio-padrão} $\sigma$ de uma variável aleatória $X$ é
definido como a raiz quadrada da variância, $\sigma(X)=\sqrt{Var
X}$.


\end{block}
\end{frame}
%
\begin{frame}{Propriedades da Variância e de outros Momentos}
%\frametitle{\textbf}
%\baselineskip=13pt
%\begin{block}{}
%
As seguintes propriedades da variância são consequências imediatas
de sua definição.
%
\begin{itemize}
\item[1.] $Var X\geq 0$.

\item[2.] Se $X=c$, $Var(X)=0$. \begin{proof} Temos que $EX=c$, logo $Var(X)=E(X-c)^2=E(0)=0$. \end{proof}

\item[3.] $Var(X+a)=Var X$, onde $a$ é uma constante real.

\begin{proof}
\begin{eqnarray}
& & Var(X+a)=E(X+a)^2-(E(X+a))^2\nonumber\\
& &=EX^2+2aEX+a^2-(EX)^2-2aEX-a^2
\nonumber\\
& &  =EX^2-(EX)^2=Var X.\nonumber
\end{eqnarray}
\end{proof}
\end{itemize}
\end{frame}
%
\begin{frame}
%\frametitle{\textbf{Propriedades da Variância e de outros Momentos}}
%\baselineskip=13pt
%\begin{block}{}
%
%
\begin{itemize}
\item[4.] $Var(aX)=a^2Var X$
\begin{proof}
\begin{eqnarray}
& & Var(aX)=E(aX)^2-(E(aX))^2\nonumber\\
& & =a^2EX^2-a^2(EX)^2=a^2Var X. \nonumber
\end{eqnarray}
\end{proof}
%
\item[5.] Se $X$ e $Y$ forem variáveis aleatórias mutuamente independentes, então $Vax(X+Y)=Var X+Var Y$.
\begin{proof}\begin{eqnarray}
& & Var(X+Y)=E(X+Y)^2-[E(X+Y)]^2 \nonumber\\
& & =E(X^2+2XY+Y^2)-(EX)^2-2EXEY-(EY)^2\nonumber\\
& & =EX^2+EY^2-(EX)^2-(EY)^2+2E(XY)-2EXEY\nonumber\\
& & =Var X+VarY\nonumber
\end{eqnarray}
%%}
\end{proof}
%
\end{itemize}
%
%\end{block}
\end{frame}
%
\begin{frame}
%\frametitle{\textbf{Propriedades da Variância e de outros Momentos}}
%\baselineskip=13pt
%\begin{block}{}
%
%
\begin{itemize}
%
%
\item[6.] Se $X_1,\ldots,X_n$ são variáveis aleatórias independentes,
então $Var(X_1+\cdots X_n)=Var X_1+\cdots +Var X_n$. Esta
propriedade segue da propriedade anterior e de uma aplicação de
indução matemática.

\item[7.] {\bf Desigualdade de Chebyshev Generalizada.} Dado um conjunto
$A$ e uma função $g(x)$ tal que $\forall x$ $g(x)\geq I_A(x)$, tem-se
que $P(X\in A)\leq \min(1,Eg(X))$.

\begin{proof} Pela monotonicidade da Esperança, temos que $Eg(X)\geq
EI_A(X)=P(X\in A)$. Mas, como a cota superior pode exceder 1, temos
que $$\min(1,Eg(X))\geq P(X\in A).$$ \end{proof}

\end{itemize}

%\end{frame}
%
%\begin{frame}
%\frametitle{\textbf{Propriedades da Variância e de outros Momentos}}
%\baselineskip=13pt
%\begin{block}{}
%
%
%
\begin{corol} Seja $X$ uma
variável aleatória, então para todo $\epsilon>0$, $P(|X|\geq \epsilon)\leq \frac{E|X|
}{\epsilon}$. \end{corol}

\begin{proof}
Escolha $A=\{x:|x|\geq \epsilon\}$ e $g(x)=\frac{|x|}{\epsilon}$. Note que $g(x)\geq I_A(x)$, então $P(|X|\geq \epsilon)\leq \frac{E|X|}{\epsilon}$.
\end{proof}
%\end{block}
\end{frame}
%
\begin{frame}
%\frametitle{\textbf{Propriedades da Variância e de outros Momentos}}
%\baselineskip=13pt
%\begin{block}{}
%
\begin{corol}
Se $Z\geq 0$ e $EZ=0$, então $P(Z=0)=1$.
\end{corol}
%
\begin{proof}
$P(Z\geq \frac{1}{n})\leq nEZ=0$. Como $[Z>0]=\union_n[Z\geq \frac{1}{n}]$, temos que
$$P(Z>0)=P(\union_n[Z\geq \frac{1}{n}])\leq \sum_n P(Z\geq \frac{1}{n})=0.$$
Portanto, $P(Z=0)=1-P(Z>0)=1$.
\end{proof}
%
Note que este último corolário implica que, quando $Var(X)=0$, ou seja $E(X-EX)^2=0$, temos que $P(X=EX)=1$, i.e., $X$ é constante com probabilidade 1.
%
%
\begin{corol}[Desigualdade (Original) de Chebyshev.] Seja $X$ uma
variável aleatória, então $P(|X-EX|\geq \epsilon)\leq {Var
X}/{\epsilon^2}$. \end{corol}

\begin{proof} Escolha $A=\{x:|x|\geq \epsilon\}$ e
$g(x)=\frac{x^2}{\epsilon^2}$. Note que $g(x)\geq I_A(x)$, então
pelo teorema anterior, $P(X\in A)=P(|X|\geq \epsilon)\leq
{EX^2}/{\epsilon^2}$. Substituindo $X$ por $X-EX$, temos
$P(|X-EX|\geq \epsilon)\leq {Var X}/{\epsilon^2}$. \end{proof}


%\end{block}
\end{frame}
%
\begin{frame}
Note que a desigualdade de Chebyshev converte conhecimento sobre um
momento de segunda ordem ou uma variância numa cota superior para a
probabilidade da cauda de uma variável aleatória.

%\frametitle{\textbf{Propriedades da Variância e de outros Momentos}}
%\baselineskip=13pt
%\begin{block}{}
%
%
\begin{itemize}
%
%
%
\item[8.] Se $X$ e $Y$ são variáveis aleatórias em $(\Omega,\A,P)$ tais que $E|X^t|<\infty$ e $E|Y^{t}|<\infty$, então $E|X+Y|^t<\infty$.
\begin{proof}
$|X+Y|\leq |X|+|Y|\leq 2\max(|X|,|Y|)$. Portanto, $|X+Y|^t\leq 2^t\max(|X|^t,|Y|^t)\leq 2^t(|X|^t+|Y|^t)$. Logo, $E|X+Y|^t\leq 2^t(E|X|^t+E|Y|^t)<\infty$.
\end{proof}

Como $E|X|^t<\infty$ obviamente implica $E|aX|^t<\infty$, $\forall a\in \IR$, esta propriedade diz que a classe de variáveis aleatórias em $(\Omega,\A,P)$ possuidoras do $t$-ésimo momento finito é um espaço vetorial ou espaço linear.

\item[9.] $Var X=E(X-\mu)^2=\min_{c\in\IR}E(X-c)^2$.

\begin{proof}
\begin{eqnarray}
& & (X-c)^2=(X-\mu+\mu-c)^2=(X-\mu)^2+2(\mu-c)(X-\mu)+(\mu-c)^2,
\nonumber
\end{eqnarray}
logo
\begin{eqnarray}
& & E(X-c)^2=E(X-\mu)^2+2(\mu-c)(EX-\mu)+(\mu-c)^2 =Var X +(\mu-c)^2.\nonumber
\end{eqnarray}
Portanto, $E(X-c)^2\geq E(X-\mu)^2, \forall c\in \IR$.
\end{proof}

\end{itemize}


\end{frame}
%
\begin{frame}{Momentos Conjuntos}
%\frametitle{\textbf}
%\baselineskip=13pt
%\begin{block}{}
%
Podemos definir a noção de momento quando lidamos com vetores aleatórios.

\begin{defi}
Seja $\vec{X}=(X_1,X_2,\ldots,X_k)$ um vetor aleatório $k$-dimensional. Então, os {\em momentos conjuntos} de $\vec{X}$ são da forma $E(\prod_{i=1}^{k}X_i^{j_i})$, onde $j_i$'s são inteiros positivos, se esta esperança existir. De forma análoga ao caso unidimensional pode-se definir também {\em momentos conjuntos centrais}.
\end{defi}
%
No caso bidimensional, temos que a correlação e a covariância são momentos conjuntos que
são medidas do grau de dependência linear entre duas variáveis.
%
\begin{defi}
A correlação entre duas variáveis aleatórias $X$ e $Y$ é dada por
$EXY$ se esta esperança existe. A covariância entre elas é dada por
$Cov(X,Y)=E[(X-EX)(Y-EY)]=EXY-(EX)(EY)$.
\end{defi}
%
%\end{block}
%\end{frame}
%%
%%\begin{frame}
%%\frametitle{\textbf{Momentos Conjuntos}}
%%\baselineskip=13pt
%%\begin{block}{}
%%
%%
Note que $Cov(X,X)=Var X$. Pela prova da propriedade 5 de variância,
vemos que a relação
$Var(X+Y)=Var X+Var Y+2Cov(X,Y)$ é válida.

Duas varáveis são {\em não-correlacionadas} se
$Cov(X,Y)=0$. Como já provamos que se $X$ e $Y$ são independentes, então
$EXY=EXEY$. Temos que se $X$ e $Y$ são independentes, elas
necessariamente são não-correlacionadas. O contrário nem sempre é
verdadeiro como o próximo exemplo ilustra.

%\end{block}
\end{frame}
%
\begin{frame}
%\frametitle{\textbf{Exemplo}}
%\baselineskip=13pt
%\begin{block}{}
%
%
\begin{exem}
Se $X$ é uma variável aleatória tal que $P(X=-a)=P(X=a)=1/2$ e
$Y=X^2$, temos que $EXY=-a^3(1/2)+a^3(1/2)=0$ e
$EX=-a(1/2)+a(1/2)=0$. Logo, $EXY=EXEY=0$, ou seja, $Cov(X,Y)=0$.
Porém, $X$ e $Y$ não são independentes, pois $Y$ é uma função de
$X$.
\end{exem}
%
%\end{block}
%\end{frame}
%
%\begin{frame}
%\frametitle{\textbf{Momentos Conjuntos}}
%\baselineskip=13pt
%\begin{block}{}
%
%
Vejamos agora uma expressão para a variância da soma de $n$ variáveis aleatórias.

\begin{teo}
Sejam $X_1,X_2,\ldots,X_n$ variáveis aleatórias tais que $Var(X_i)<\infty$, então
$$Var(X_1+\cdots+X_n)=\sum_{i=1}^{n}Var X_i +2\sum_{i<j}Cov(X_i,X_j).$$
\end{teo}

\begin{proof}\vspace{-0.3cm}
\begin{eqnarray}
& & Var(X_1+\cdots+X_n)=E(X_1+\cdots+X_n-E(X_1+\cdots+X_n))^2 \nonumber\\
& & =E(\sum_{i=1}^{n}(X_i-EX_i))^2=E[\sum_{i=1}^{n}(X_i-EX_i)^2+2\sum_{i<j}(X_i-EX_i)(X_j-EX_j)] \nonumber\\
& & =\sum_{i=1}^{n}Var(X_i)+2\sum_{i<j}Cov(X_i,X_j). \nonumber
\end{eqnarray}
\end{proof}
\end{frame}
%
\begin{frame}
%\frametitle{\textbf{Momentos Conjuntos}}
%\baselineskip=13pt
%\begin{block}{}
%
%
\begin{corol}
Sejam $X_1,X_2,\ldots,X_n$ variáveis aleatórias tais que $Var(X_i)<\infty$ e $Cov(X_i,X_j)=0$ para $i\ne j$, então
$$Var(X_1+\cdots+X_n)=\sum_{i=1}^{n}Var X_i.$$
\end{corol}
%
O próximo teorema trata de uma importante desigualdade em teoria da
probabilidade:

\begin{teo} $$(E(XY))^2\leq EX^2EY^2\mbox{ e }(Cov(X,Y))^2\leq Var XVar Y.$$
\end{teo}
%
%%\end{block}
%\end{frame}
%%
%%\begin{frame}
%%\frametitle{\textbf{Prova do Teorema}}
%%\baselineskip=13pt
%%\begin{block}{}
%%
%%
%%\prv 
\begin{proof}
$(aX+Y)^2\geq 0\Rightarrow E(aX+Y)^2\geq 0\Rightarrow
a^2EX^2+2aEXY+EY^2\geq 0$. Observe que esta equação do segundo grau
em $a$ não pode ter duas raízes reais diferentes, pois caso
contrário essa expressão seria negativa para os valores entre as
raízes. Então, utilizando a regra do discriminante, temos que
$$4(EXY)^2-4EX^2EY^2\leq 0,$$
e temos a primeira desigualdade. A segunda desigualdade segue da
primeira trocando $X$ por $X-EX$ e $Y$ por $Y-EY$ na expressão da
primeira desigualdade. 
\end{proof}
\end{frame}
%
\begin{frame}{Coeficiente de Correlação}
%\frametitle{\textbf}
%\baselineskip=13pt
%\begin{block}{}
%
%
O {\em coeficiente de correlação} entre duas variáveis aleatórias $X$ e $Y$ é dado por
$$\rho(X,Y)={Cov(X,Y)}/{\sqrt{Var(X)Var(Y)}}.$$
O teorema anterior provou que $|\rho(X,Y)|\leq 1$. O próximo teorema mostra que o módulo do coefieciente de correlação entre duas variáveis é igual a 1 se, e somente se, as variáveis são linearmente dependentes.
%
\begin{teo}
Sejam $X$ e $Y$ variáveis aleatórias com variâncias finitas e positivas. Então,
\begin{enumerate}
\item[(a)] $\rho(X,Y)=1$ se, e somente se, $P(Y=aX+b)=1$ para algum $a>0$ e $b\in \IR$.
\item[(b)] $\rho(X,Y)=-1$ se, e somente se, $P(Y=aX+b)=1$ para algum $a<0$ e $b\in \IR$.
\end{enumerate}
\end{teo}
%
%%\end{block}
%\end{frame}
%%
%%\begin{frame}
%%\frametitle{\textbf{Prova do Teorema}}
%%\baselineskip=13pt
%%\begin{block}{}
%%
%%
%%\prv
\begin{block}{Demonstração}
Parte (a). Como $(({(X-EX)}/{\sqrt{Var(X)}})-({(Y-EY)}/{\sqrt{Var(Y)}}))^2\geq 0$, temos que
\begin{eqnarray}
& & 0\leq E(\frac{X-EX}{\sqrt{Var(X)}}-\frac{Y-EY}{\sqrt{Var(Y)}})^2  = E(\frac{X-EX}{\sqrt{Var(X)}})^2+E(\frac{Y-EY}{\sqrt{Var(Y)}})^2\nonumber\\
& & -\frac{2}{\sqrt{Var(X)Var(Y)}}E((X-EX)(Y-EY)) \nonumber \\
& & =\frac{Var X}{Var X}+\frac{Var Y}{Var Y}-\frac{2Cov(X,Y)}{\sqrt{Var(X)var(Y)}}=2-2\rho(X,Y). \nonumber
\end{eqnarray}
\end{block}
\end{frame}
%
\begin{frame}
%\frametitle{\textbf{Prova do Teorema}}
%\baselineskip=13pt
\begin{block}{}
%
Se $\rho(X,Y)=1$, então
$$E(\frac{X-EX}{\sqrt{Var(X)}}-\frac{Y-EY}{\sqrt{Var(Y)}})^2=0,$$
o que por sua vez implica que
$$P(\frac{X-EX}{\sqrt{Var(X)}}=\frac{Y-EY}{\sqrt{Var(Y)}})=1,$$
em outras palavras,
$$P(Y=EY+\frac{\sqrt{Var Y}}{\sqrt{Var X}}(X-EX))=1.$$

A prova da parte (b) é análoga, substituindo o sinal ``+'' por ``-'' na expressão acima. Verifique os detalhes!
\end{block}
\end{frame}
%
\begin{frame}{Desigualdade de Hölder}
%\frametitle{\textbf}
%\baselineskip=13pt
%\begin{block}{}
%
%
%
O próximo teorema apresenta uma nova relação entre momentos conjuntos de variáveis aleatórias. Ele é conhecido como {\em Desigualdade de Hölder}.
%
\begin{teo}
Suponha que $p$ e $q$ satisfazem: $p>1$, $q>1$, e $\frac{1}{p}+\frac{1}{q}=1$. Então, se $E(|X|^p)<\infty$ e $E(|Y|^q)<\infty$, temos que
$$E(|XY|)\leq (E|X|^p)^{1/p}(E|Y|^q)^{1/q}.$$
\end{teo}
%
%\end{block}
%\end{frame}
%
%
%\begin{frame}
%\frametitle{\textbf{Prova do Teorema}}
%\baselineskip=13pt
%\begin{block}{}
%
%
%\prv
\begin{block}{Demonstração}
A prova da desigualdade de Hölder utiliza um argumento de convexidade. Como $|X|^p\geq 0$ (resp., $|Y|^q\geq 0$), já vimos que se $E|X|^p=0$, então $P(X=0)=1$. Portanto, em ambos os casos $E(|XY|)=0$ e a desigualdade de Hölder é válida. Considere então o caso em que o lado direito da desigualdade de Hölder é estritamente positivo.

Note que para $a>0$ e $b>0$, existe $s,t\in\IR$ tal que
$$a=\exp(\frac{s}{p})\mbox{ e }b=\exp(\frac{t}{q}).$$
Como a função exponencial é convexa e $p^{-1}+q^{-1}=1$, temos por convexidade que
$$\exp(\frac{s}{p}+\frac{t}{q})\leq p^{-1}\exp(s)+q^{-1}\exp(t),$$

\end{block}
\end{frame}
%
\begin{frame}
%\frametitle{\textbf{Prova do Teorema}}
%\baselineskip=13pt
\begin{block}{}
	ou pela definição de $s,t$
	$$ab\leq p^{-1}a^p+q^{-1}b^q.$$


Agora substituindo $a$ por $\frac{|X|}{(E(|X|^p))^{1/p}}$ e $b$ por $\frac{|Y|}{(E(|Y|^q))^{1/q}}$, temos
\begin{eqnarray}
& & \frac{|XY|}{(E(|X|^p))^{1/p}(E(|Y|^q))^{1/q}}\nonumber\\
& & \leq p^{-1}(\frac{|X|}{(E(|X|^p))^{1/p}})^p+q^{-1}(\frac{|Y|}{(E(|Y|^q))^{1/q}})^q.\nonumber
\end{eqnarray}
Finalmente, tomando o valor esperado, temos
\begin{eqnarray}
& & \frac{E|XY|}{(E(|X|^p))^{1/p}(E(|Y|^q))^{1/q}} \nonumber\\
& & \leq p^{-1}(\frac{E|X|^p}{(E(|X|^p))})+q^{-1}(\frac{E|Y|^q}{(E(|Y|^q))})\nonumber\\
& &  \leq p^{-1}+q^{-1}=1.\nonumber
\end{eqnarray}
%\eprv
\end{block}
\end{frame}
%
\end{document}

